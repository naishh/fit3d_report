% DRAFT
\section{Preliminaries on Computer Vision}

% todo do ordering of subsections
\subsection{Edge detection}
\label{sec:edge}
%todo

\subsection{Hough transform}
\label{sec:prelimHough}
\subsubsection{Theory}
	A widely used method for extracting line segments is the Hough transform
	\cite{Hough}.
	In the Hough transform, the main idea is to consider the characteristics of a
	straight line not as its image points $(x1, y1)$, $(x2, y2)$, but in
	terms of the parameters of the straight line formula $y = mx + b$. i.e., the
	slope parameter $m$ and the intercept parameter $b$.


	The Hough transform transforms the line $y = mx + b$ 
	to a point $(b,m)$ in parameter space.
	With this representation it is impossible to describe a vertical line as 
	the slope $m$ is infinite.
	Therefore it is better to use a different pair of parameters, denoted $r$ and $\theta$ .  These are the Polar Coordinates.

	The parameter $r$ represents the distance between the origin and the line  and $\theta$ is the angle of the vector orthogonal to the line.
	Using this parameterization, the equation of the line can be written as

\[	    y = \left(-{\cos\theta\over\sin\theta}\right)x + \left({r\over{\sin\theta}}\right) \]

	\[r = x \cos \theta+y\sin \theta \]
	
	This means that a straigth line in $(x,y)$ space appears as a sinusoidal
	curve in the Hough parameter $(r,\theta)$ space.  Let's see an example, the
	following image is transformed into the space $(r,\theta)$.
	
	%todo introduce the accumulator array,
	%todo transform gif to eps 
	\fig{HoughTransform_edge.eps}{An input image, consisting of eight straight lines, for the Hough transform}{0.5}
	\figsHor{HoughTransform_peaks}{HoughTransform_peaks1.eps}{HoughTransform_peaks2.eps}{Hough transform} {$(r, \theta)$ values}{$(r, \theta)$ accumulator array (quantized)}

	 
	 As you can see for every edge point 
	 in Figure \ref{fig:HoughTransform_edge.eps} 
	 a curve is generated in $(r,\theta)$ space in Figure 
\ref{fig:HoughTransform_peaks1.eps}.
	 On eight positions (white dots) the number of intersecting sinusoidal
	 curves is high, these position correspond to the eight separate straight
	 line segments in Figure \ref{fig:HoughTransform_edge.eps} .
	
	This makes the problem of detecting straigth lines in finding peaks in the 
	Hough parameter $(r,\theta)$.

\subsubsection{Implementation}
	The input of a Hough transform is a binary image, in our research it is the output of 
	the skyline detector (\ref{sec:skylinedetection}) or in the case of window
	detection (\ref{sec:windowDetection}) the output of an edge image.\\

	The Houghtransform develops an accumulator array of a quantized parameter space $(r, \theta)$.

	It loops through the binary image and for each positive value 
	it generates all possible lines, quantized $(r, \theta)$ pairs, that intersect with this point.
	For each candidate it increases a vote in the accumulator array.
	Lines $(r, \theta)$ that receive a large amount of votes
	i.e. the dots in Figure \ref{fig:HoughTransform_peaks1.eps} are the found straigth lines in the $(x,y)$ space.

	These positions are found by looking for local maxima in the accumulator array.

\subsubsection{$\theta$ constrained Hough transform}
The accumulator array consist of two dimensions $r$ and $\theta$.
$\theta$ typically ranges from $[-90..90)$ resulting in 180 unique bins.
Note that a line with $(r, \theta) = (t,j)$ can also be represent by the identical $(-t, j-90)$, e.g. $(4, 135)$ == $(-4, 45)$.
This makes it possible to represent every line with the interval $[-90..90)$.\\

Sometimes we want to find lines that have a certain angle.
For example the skyline of a building will appear about horizontal, furthermore with window detection one would like to detect edges in the horizontal and vertical directions.
This can easily achieved by adjusting the $\theta$ range.
For example if one would detect lines in the horizontal direction of
a photograph of a building taken by a user, $\theta = [-10..0..10]$.Although
only $\theta = 0$ presents an exact horizontal line we broaden the interval
because the user hardly ever holds the camera exactly orthogonal.
	
\subsubsection{Matlab parameters}
We used a standard Matlab implementation of the Hough transform for straight lines.  This implementation comes with some interesting parameters:\\

	The \emph{minimum length} parameter specifies the minimum length that a line must have to be valid. This is especially interesting if we want to detect a large straight skyline or if we want to discard lines that are to small to form for example a window.\\

	Furthermore it contains the parameter \emph{FillGap} that specifies the distance
	between two line segments associated with the same $(r, theta)$ pair
	When this inter line segment distance is less then the \emph{FillGap} parameter, it merges the line segments into a single line segment. In our application this parameter is of particular interest when we want to merge lines that are interrupted by for example an occluding tree or street lamp.\\


\subsubsection{Other shapes}
The Hough transform isn't bound to finding straight lines, in fact it could be
applied to any space which can be represented by a set of parameters.  For
instance, a circle can be transformed into a set of three parameters, $(x,y,r)$
representing its center and radius, this makes the Hough space three
dimensional. 
For more complicated shapes, the 
\emph{Generalised Hough transform} is used.
It uses a feature that votes for a particular position, orientation and/or scaling of a certain predefined shape.  The shapes are predefined in a look-up table.


% todo
%\subsubsection{Complexity of Hough transform}



\subsection{Transform coordinates to a line equation}
\label{sec:lineeq}
%TODO uitleggen hoe je van 2 coords een lijnvergelijking maakt
\TODO{, explain 2 coords to an line equation}


\subsection{Coordination systems}
The camera can be described in the same coordinate system as the 3D point we are
looking for, we call this the world coordinate system. 
The camera center is a value in $\mathbb{R}$3 that represents the camera's position in the world
coordinate system. If the camera rotates, it rotates around this point.
\TODO{ elaborate}\\

\subsection{Homogenous coordinate}
\TODO{ explain how to add extra homogeneous dimension (x,y,1) and why this is}


\subsection{Planes and walls}
\label{sec:planeswalls}
\TODO{ How to span plane by wall coords}

\subsection{Getting the camera centers and viewing directions}
\label{sec:cameracenters}
\TODO{ Explain here about fit3d toolbox}

\subsection{Calibrating the camera with the Bouguet toolbox}
\label{sec:bouguet}
TODO
\TODO{TODO see ref 15 and fit3d toolbox p4}

	%\paragraph{Getting the calibration matrix}
	%TODO inspriation
	% isaacs paper 
	% google 
	% matlab help bouget 
	%
\subsection{Axis angle representation}
\label{sec:axisAngle}
\TODO{Axis angle representation, wikipedia}
