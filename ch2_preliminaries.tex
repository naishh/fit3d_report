% DRAFT
\section{Preliminaries on Computer Vision}

% todo do ordering of subsections
\subsection{Edge detection}
\label{sec:edge}
%todo

\subsection{Hough transform}
	A widely used method for extracting line segments is the Hough transform
	\cite{Hough}.
	In the Hough transform, the main idea is to consider the characteristics of a
	straight line not as its image points $(x1, y1)$, $(x2, y2)$, but in
	terms of the parameters of the straight line formula $y = mx + b$. i.e., the
	slope parameter $m$ and the intercept parameter $b$.\\
	The Hough transform transforms the line $y = mx + b$ 
	to a point $(b,m)$ in parameter space.
	E.g. the following images is transformed into the Hough $(r,\theta)$ parameter space.
	
	%todo introduce the accumulator array,
	%todo transform gif to eps 
	\fig{HoughTransform_edge.eps}
	\fig{HoughTransform_peaks1.eps}
	\fig{HoughTransform_peaks2.eps}
	 
	 As you can see the eight separate straight line segments in \ref{fig:HoughTransform_edge.eps} 
	 a curve is generated for 
	 in polar space for each edge point in cartesian space. 
	 
	 The accumulator array, when viewed as an intensity image, looks like 

	% todo image of hough peaks matlab manual
	


	%todo include r and theta

	% secundairy todo add other paramater space
	% also explain why we use the angle parameter space
	% to devide the lines


\subsubsection{Direction ordering}
% explain  something like:
% From the edge image we extract two different groups of Houghlines, horizontal and 
% vertical.  We set the $\theta$ bin ranges in the Hough transform that control the
% allowed angles of the Houghlines to extract the two groups. The horizontal group
% has a range of [-30..0..30] degrees, where 0 presents a horizontal line. The vertical



% HOUGH TRANSFORM komt hier
\subsection{Fit3d toolbox}
The fit3D toolbox \cite{Fit3d} 
TODO explain structure by motion etc.

\subsection{Transform coordinates to a line equation}
\label{sec:lineeq}
%TODO uitleggen hoe je van 2 coords een lijnvergelijking maakt
TODO, explain 2 coords to an line equation


\subsection{Coordination systems}
The camera can be described in the same coordinate system as the 3D point we are
looking for, we call this the world coordinate system. 
The camera center is a value in $\mathbb{R}$3 that represents the camera's position in the world
coordinate system. If the camera rotates, it rotates around this point.
TODO elaborate\\

\subsection{Homogenous coordinate}
TODO explain how to add extra homogeneous dimension (x,y,1) and why this is


\subsection{Planes and walls}
\label{sec:planeswalls}
TODO How to span plane by wall coords

\subsection{Getting the camera centers and viewing directions}
\label{sec:cameracenters}
TODO Explain here about fit3d toolbox

\subsection{Calibrating the camera with the Bouguet toolbox}
\label{sec:bouguet}
TODO
%TODO see ref 15 and fit3d toolbox p4

	%\paragraph{Getting the calibration matrix}
	%TODO inspriation
	% isaacs paper 
	% google 
	% matlab help bouget 
	%

