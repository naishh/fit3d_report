%TODO ground plane extraction together with skyline gives facade region

\section{Skyline detection}
\label{sec:skylinedetection}
 \subsection{Introduction}
If we take a regular image on which both sky and earth are present, there
is often a clear separation between them. This separation is called the
skyline. %oehhhhhhh  dat is echt masterlijk ouwe
The detection of this skyline has proven to be a very successful computer vision
application in a wide range of domains ranging from object detection, 
guiding micro air vehicles, car localization, etc. 

In this research skyline detection is applied on different views of a scene to
estimate heights of facades of a 3D model. This is a novel application for
skyline detection.\\

The organization of this chapter is as follows: first we give a summary of
related work on skyline detection. Next we explain how we developed a new
robust skyline detection algorithm. Then we present and discuss some results
and, finally, conclusions are given.

\subsection{Related work}
Castano et al. \cite{Dust} present a clear introduction of different skyline
detection techniques. 

\paragraph{Detection of dust devils and clouds on Mars}
In \cite{Dust}, mars Exploration Rovers are used to detect clouds and dust devils on Mars.
Their approach is to first identify the sky and then determine if there are
clouds in the region segmented as sky. The sky is detected by an innovative
algorithm that consists of three steps. First they place seeds in a sliding
window whenever the homogeneity of the window is high. Then they grow this seeds
in the direction of edges which are estimated using a Sobel edge detector.
Finally each pixel located above the grew seeds is classified as sky.\\

This seed growing method looks like the a very sophisticated one, as it is
accurate and autonomous. However, in our research we have a stable scene with
sharp edges at the building contour so this method would be an implementation
overkill. 


\paragraph{Horizon detection for Unmanned Air Vehicles}
In this domain \cite{Guidedflight}, scientists detect the horizon to stabilize and control the
flight of Unmanned Air Vehicles.\\  
S.M. Ettinger et all \cite{Guidedflight} use a horizon detector that takes
advantage of the high altitude of the vehicle, in that way the horizon is
approximated to be a straight line. 
This straight line separates the image into sky and ground. They use color as
a measure of appearance and generate two color distributions: one for the sky
and one for the ground. They use the covariance and the eigen values of the
distributions to guide a bisection search for the best separation. The line that
best separates the two distributions is determined to be the skyline.\\

This work is not applicable for detecting a building contour as the straight
line assumption doesn't work. But it needs to be mentioned that some ideas for
section \ref{extractinglinesegments} are inspired by this method.

\paragraph{Planetary Rover localization}
Cozman et al. \cite{Rover} use skyline detection in planetary rovers to estimate 
their location. 
To recover the rover's position they match image structures with a given map
of the landscape (hills, roads, etc) and align both images.
The matching process was first based on feature matching. In an improved version
the matching process was done by searching for correspondences among dense
structures in the image and on the given map, so called signal based matching.\\
The advantage of their algorithm is the simplicity and effectiveness, this
could make their algorithm suitable for this project. A big drawback is that
they prefer speed over accuracy. To increase accuracy, the detector is part
of an interactive system where an operator refines the skyline. For our
application the skyline detector must operate without any user interaction.
This brings us to our research question.

\subsection{Method}
\paragraph{Research question}
Can we build a skyline algorithm that operates without any user interaction, is
simple, is fast, and yet accurate enough to provide a solid skyline that can be
used to estimate wall heights?

\subsubsection{Situation and assumptions}
As the Rover method \cite{Rover} is simple and effective we used it as a basis and 
build a custom algorithm with higher accuracy on top of that. 
\paragraph{Situation}
Before we present the method, we define the situation and make some
assumptions.\\

\paragraph{Definition: skyline in urban scene}
\emph{A skyline in an urban scene is a set of points of the size $w$ (where $w$ is the
width of the image) where each point describes the location of the
transition from the sky to an object (e.g. a building) which is connected to the
earth.}\\

How are we going to detect this sky-building transition point?\\ 
In general, the color of the sky is very different than the color of the
building. A color-based edge detector would be an intuitive decision as this
produces edges on regions where intense  color transitions appear. However, the
sky and the building itself also contains color transitions (caused by for
example clouds and windows). So how do we determine the right transition
(edge)?\\

One of the solutions is to increase the threshold of the edge detector. In this
way the detector will only return intense color transitions. Note that this will
only pay off if the building-sky transition is the biggest transition in the image. 
Its easy to see that this is a tricky assumption as other objects may contain
sharper color transitions. Furthermore it would not be robust to a change in
the lightning conditions, influenced heavily by the weather.\\

To solve this problem we draw an assumption that is based on the
idea of \cite{Rover}. Instead of using the sharpest edge we take the most upper sharp
edge and classify this edge as the skyline.\\

\paragraph{Top sharp edge assumption}
\emph{The first sharp edge (seen from top to bottom) in the image 
represents the skyline.}

Having defined the situation and assumptions we now explain our algorithm.

\subsubsection{Related algorithm}
As our algorithm is based on a related algorithm presented in \cite{Rover},
this is described first.

% nog meer woordjes vooraf?
%gaussian smooth (explain if reader is thumb)

The algorithm uses three main steps first it applies a smoothing preprocessing
step then it calculates the intensity gradient to find a big color transformation
and finally it searches for the highest transformation.

The preprocessing step is used to increase the difference between sharp and
vague edges, and to let sharp edges stand out more and vague edges disappear.
The preprocessing consists of a Gaussian smoothing operator on the input image.
Next the smoothed image is sliced in \#$w$ pixel columns. Each column represents
the values of a discretized function. These values are transformed to their
derivative, called the smoothed intensity gradient. The values
of this column are high when a big change in color happens (e.g. an edge is
detected) at that location on the image. Next the system walks through the
values of a column, starting from the top. When it detects a value with a
gradient higher then a certain threshold it stores its y-position (the height)
and continues to the next column. After the position in each column of the
highest sharp edge is determined the algorithm is done. The result is a set of
$y$ coordinates of length $w$, that represent the skyline. 

\subsubsection{Improved algorithm}
Taking the smoothed intensity gradient is a computational cheap way to
detect edges. It also has a big disadvantage because it is not robust to vague edges
(they don't survive the threshold). It is not surprising that the algorithm in \cite{Rover}
was used in an interactive system where the user has to refine the result.\\

Our aim is to develop an autonomous skyline detector, the only user interaction
that we allow is to provide the system some parameters. Furthermore the vague
edges need to be detected if they are part of the skyline. We will now discuss
the adaptations that we developed with respect to the related algorithm.\\

The column based approach of the related algorithm seems to be very useful and is
therefore unchanged. To be robust to vague edges we explored and tested
edge detecting types that are different then the smoothed intensity gradient
based method.\\

\label{sec:edgeDet}
The output of the different edge detection techniques was studied on an empirical
basis and the Canny edge detector \cite{Canny} was a clear winner. This is
probably because Canny is a more advanced edge detector. It uses two
thresholds, one to detect strong and one to detect weak edges. It includes the weak edges in the
output, but only if they are connected to strong edges. In Table \ref{tab:edge} 
we list Matlabs built-in edge detectors together with the method explanation.

In the section \ref{sec:ResultEdge} one can find the results of the different
edge detection methods.


\begin{table}[ht]
\caption{Different edge detectors explained, Source: Matlab Documentation}
\label{tab:edge}
%note naar mijzelf altijd eerst caption dan label, dan en slechts dan gaat hte
%goed met de nummering
\begin{tabular}{|l|p{10cm}|}
	\hline
	Name & method\\
	\hline
	\hline
	Sobel					& The Sobel method finds edges using the Sobel
	approximation to the derivative. It returns edges at those points where the
	gradient of the image is maximum.\\
	\hline
	Prewitt					& The Prewitt method finds edges using the Prewitt
	approximation to the derivative. It returns edges at those points where the
	gradient of the image is maximum.\\
	\hline
	Roberts					& The Roberts method finds edges using the Roberts
	approximation to the derivative. It returns edges at those points where the
	gradient of the image is maximum.\\
	\hline
	zero-cross				& The zero-cross method finds edges by looking for zero
	crossings after filtering the image with a filter that has to be specified.\\
	\hline
	Laplacian				& The Laplacian of Gaussian method finds edges by
	looking for zero crossings after filtering the image with a Laplacian of Gaussian
	filter.\\
	\hline
	Canny					& The Canny method finds edges by looking for local
	maxima of the gradient of the image. The gradient is calculated using the derivative of
	a Gaussian filter. The method uses two thresholds, to detect strong and weak
	edges, and includes the weak edges in the output only if they are connected to
	strong edges. This method is therefore less likely than the others to be fooled
	by noise, and more likely to detect true weak edges.\\
	\hline
\end{tabular}
\end{table}
We classify Canny as the most robust edge detector, and plug it into the skyline detection
algorithm: the Canny edge detector outputs a binary image, therefore the column inlier
threshold is set to 1, which means that it finds the first pixel that is white. 
This is, as in the related algorithm, done from top to bottom for every column in
the image.\\
Because we know we are looking for sharp edges, we improved the algorithm by
introducing two preprocessing steps. First the contrast of the image is
increased, this makes sharp edges stand out more. Secondly the image undertakes
an extra Gaussian blur, this removes a large part of the noise. 

The system now has several parameters which have to be set manually by the user:
\begin{itemize}
	\item Contrast
	% officially i don't do this contrast thing ghehe, whoepsie daisy fooling the
	% reader
	\item Intensity (window size) of Gaussian blur
	\item Edge detector threshold
\end{itemize}

If the user introduces a new dataset these parameters need to be configured
as the image quality and lightning condition are scene depended.


\subsection{Results}
\subsubsection{Edge detection}
\label{sec:ResultEdge}
\fig{e_floriande_canny_050.eps}{Edge detection results. Method: Canny}{0.45}
The edge detection results of the other methods can be found in the appendix
(\ref{app:edge}).
\clearpage


\subsubsection{Skyline detection}
\paragraph{Datasets}
The skyline detection algorithm was applied on three datasets. 

\begin{table}[ht]
\caption{Dataset properties}
\begin{tabular}{|l||l|l|l|}
\hline
Name 	& Resolution 	& Source	& Location\\
\hline
\hline
Floriande & 1728x1152px  & FIT3D \cite{FIT3D}  	& Unknown\\
\hline
Bram	 & 3072x2304px  & Author					& Amsterdam, 'Postjesweg'\\
\hline
Anne	& 3072x2304px  & Author					& Amsterdam, 'Van Spilbergenstraat'\\
\hline
\end{tabular}
\end{table}

The output of the skyline detector on the \emph{Floriande}
dataset \cite{FIT3D} can be seen in Figure \ref{fig:outputSkylineIm3-3.eps}.
We show the effect of different thresholds of the edge detector on the
\emph{Anne} dataset (Figure \ref{fig:outputSkylineSpil-Im1-thresh070.eps})
and the 
\emph{Bram} dataset (Figure \ref{fig:outputSkylineSpil-Im13-thresh030.eps}). 
\newpage
\fig{outputSkylineIm3-3.eps}{The output of the skyline detector. The skyline elements are marked red.}{0.45}
%\fig{outputSkylineSpil-Im1-thresh070.eps}{The output of the skyline detector with a too high threshold (0.70)}{0.3}
\fig{outputSkylineSpil-Im13-thresh030.eps}{A scene which violates the top sharp edge assumtion. The hanging streetlight causes the detection of a sharp edge above the building. This results in a damaged skyline.}{0.3}
\clearpage

\subsection{Discussion}  % (What do my results mean to me and why)
Consider Figure \ref{fig:outputSkylineIm3-3.eps}, the largest part of the
building edge is detected. This is a good result, given the algorithm
operates without any user interaction.\\

We assumed that the first sharp edge (seen from top to bottom) in the image 
represents the skyline. We showed in Figure \ref{outputSkylineSpil-Im13-thresh030.eps}
that the first sharp edge is not always the skyline.
This holdso for more scenes: e.g. Amsterdam is full of
hanging streetlights.  Furthermore, other sharp edged objects that appear above
the building, e.g. an tree or aircraft, will also produce a scene that is not valid according the top sharp
edge assumption.  Therefore it would be nice to relax or discard the first sharp edge assumption.
This implies that we have to extend the column based approach. This is described in Future research
(\ref{sec:skylinefut}).

\subsection{Conclusion}
Let's answer our research question.
\paragraph{Research question}
Can we build a skyline algorithm that operates without any user interaction, is
simple, is fast, and yet accurate enough to provide a solid skyline that can be
used to estimate wall heights?\\

Beside some scene dependent parameters (like the threshold of the skyline) the
system works without any user interaction. No manual refinement step is needed
because the algorithm is robust and accurate enough to provide a base for the
next module in the system. Furthermore the algorithm is simple and has a low
complexity. 

It is interesting to point out that the skyline detector is a stand alone method
and it can be optimized individually without any knowledge of the other modules
of the project.  Some ideas for future research are shared next.

\subsection{Future research}
\label{sec:skylinefut}
\subsubsection{Automatic thresholding}
As the threshold is manual and scene dependent, a method for automatic
thresholding is desired. There are many studies on this topic, most of the
methods are based on the statistical analysis of the image. Detailed literature
research needs to be done and an implementation must be made to provides a value
for the threshold. If this is done the system would be 100\% automatic.

\subsubsection{Hypothesis based skyline detection}
%todo
%-----
situtaion 
the first top sharp edge 
in Figure \ref{outputSkylineSpil-Im13-thresh030.eps}
is not the skyline but a hanging streetlamb.
If we take a look at the edge image in Figure
%todo edge image!
we observe that the skyline is the second
%todo realy?
sharp edge.
Let's relax the first top sharp assumption to an assumption where the skyline is
part of the first $n$ sharp edges (seen from top) with for example $n=3$.
Now the scene doesn't violate the assumption.
The column based approach is now used to generate $n$ hypotheses and we can build
an algorithm on top of that to classify the right sharp edge as skyline.

The hypothesis classifier could be based on color or height variation.

One could extract the color distribution of the color in a certain area above
A and under U the skyline. 
A and U should 1) have a significant difference (as it is building and sky) 
and 2) should not vary much from close skyline points.

For example the first sharp edge at the streetlamp has about the same color distribution
above and under the skyline and differs from its close skyline points.
Therefore it will not be classified as a valid skyline.

height variation.



If we take a look at the related work we observe that most work is based on
detecting parts that are classified as sky and parts that are classified as
ground. The idea of detecting the sky and ground could be replaced by detecting
the sky and the building. We could use color to discriminate the building from
the sky. Different color models could be used (HSV, normalized RGB, etc.).
Because the buildings exists of repeating bricks, texture would also be a good
discriminator. We could apply the existing column based approach by using the
highest pixel (for every column) that is classified as a building pixel.\\

The results could be refined by applying detailed edge detection in the region of
the estimated skyline. In this way the outliers (e.g. the streetlight and the
tree) could be detected, no secondary outlier removal procedure is needed and
the skyline detector would be very robust.




