% thinks to bear in mind
% reader must be able to reproduce it

\section{Augmenting the 3D building estimate}
\subsection{Introduction}
From the previous we saw how every point on the skyline was projected on its most likely wall. (what does this mean in general)
We would like to use this information to augment the basic 3d model.

What is the input
what are the assumptions and heuristics
what is the output

To do this there are several steps:
(ch abstract)
First of all the individual pixels are grouped into line segments using an approach of (TODO DAVID??OFZO) Hough.
This introduces (TODO our first?) assumption 
Every line segment in 3D belongs to a specific wall which is heuristicly determined in the second step.
The third step is to combine the projected houghlines which produces and estimate of the height of each wall.
In the last step this height is used to augment the rough 3d model.


For simplicity reasons we assume that building consist of a flat roof, note that the walls may have different heights but the roof should be flat.
%TODO cons pros of this assumption
%TODO vet goed dat ik dus mijn algo different wallheights aan kan, promoten!

%this method can be seen as some sort of outlier removal?
% write someting about the happy little DAMNED trees that we need to remove hehehe

%TODO what are the mathematics

The next step 
check if we can fit lines 
%why fit lines on 2d and not in 3d
%cheaper easyer?


skyline was projected on the building 
[TODO img]
can use 

So we have a bunch 
segments of the skyline 
because the are straight
building outline

This section describes how bla bla

% chaining
The input of the algo comes from the previous step
a bunch of line segments in 3d
The output is an updated 3d model 




\subsection{The algorithm}
\subsubsection{Houghlines}
	chainging / motivation:


	%houghlines on edge detection
	% write this chapter as last because
	% use wikipedia and my previous research for it 
\subsubsection{Houghline wall recognition}
	%decide which line segment represents which wall (heuristic)
\subsubsection{Wall height estimation}
	%calc average height of lines for each wall
	% update belonging wall




endpoints of houghlines belong to certain wall
example images!


Plan:
write thesis thing without the heuristic part

IDEAS:
heuristic 1:
take a point in the middle and let it decide

heuristic 2:
the endpoint that lies more in the middle decides which wall we take
this can be calculated using the difference of distances to the wall corner points

heuristic 3:
take points on a houghline
make a rating that the more in a middle a point is the more evidence there is
this is because it is more away from a corner so lower change of other wall

idea:
program and test different heuristics?
do analytic test (compare 1 and 3 analyticly)
	to much detail for something like this? or not?


KLAD
every wall of building specific height


wat is the input
output

found houghlines transformed in to 3d
a bunch of lines in 3d where some
projected houghlines


% TODO
talk about
what to do with outliers

\subsection{Results (What did I find)}
\subsection{Discussion (What do my results mean to me and why)}
\subsection{Conclusion and Future work}
The flat roof assumption could be stretched to a roof consist of two planes
%TODO write something about complications and wat je er aan hebt



