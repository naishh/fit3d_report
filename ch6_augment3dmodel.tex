% thinks to bear in mind
% reader must be able to reproduce it

\section{Augmenting the 3D building estimate}
\subsection{Introduction}


From the previous... 

So we have a bunch 
segments of the skyline 
because the are straight
building outline

This section describes how bla bla

% chaining
The input of the algo comes from the previous step
a bunch of line segments in 3d
The output is an updated 3d model 




\subsection{The algorithm}
\subsubsection{Houghlines}
	chainging / motivation:


	%houghlines on edge detection
	% write this chapter as last because
	% use wikipedia and my previous research for it 
\subsubsection{Houghline wall recognition}
	%decide which line segment represents which wall (heuristic)
\subsubsection{Wall height estimation}
	%calc average height of lines for each wall
	% update belonging wall




endpoints of houghlines belong to certain wall
example images!


Plan:
write thesis thing without the heuristic part

IDEAS:
heuristic 1:
take a point in the middle and let it decide

heuristic 2:
the endpoint that lies more in the middle decides which wall we take
this can be calculated using the difference of distances to the wall corner points

heuristic 3:
take points on a houghline
make a rating that the more in a middle a point is the more evidence there is
this is because it is more away from a corner so lower change of other wall

idea:
program and test different heuristics?
do analytic test (compare 1 and 3 analyticly)
	to much detail for something like this? or not?


KLAD
every wall of building specific height


wat is the input
output

found houghlines transformed in to 3d
a bunch of lines in 3d where some
projected houghlines


% TODO
talk about
what to do with outliers

\subsection{Results (What did I find)}
\subsection{Discussion (What do my results mean to me and why)}
\subsection{Conclusion and Future work}


