\documentclass[10pt]{article}
\usepackage{graphicx, subfigure}

\usepackage{amsmath} % for the argmin
\newcommand{\argmax}[1]{\underset{#1}{\operatorname{argmax}}}
\newcommand{\argmin}[1]{\underset{#1}{\operatorname{arg\ min}}}
\setlength{\parindent}{0in}
\newcommand{\code}[1]{\texttt{#1}}

\newcommand{\fig}[3]{
	\begin{figure}[!ht]
	\centering
	\includegraphics[scale=#3]{img/#1}
	%\includegraphics[width=420px]{img/#1}
	\caption{#2}
	\label{fig:#1}
	\end{figure}
}
\newcommand{\fignocaption}[3]{
	\begin{figure}[!ht]
	\centering
	\includegraphics[scale=#3]{img/#1}
	\label{fig:#1}
	\end{figure}
}

\newcommand{\figw}[3]{
	\begin{figure}[!ht]
	\centering
	\includegraphics[width=#3]{img/#1}
	\caption{#2}
	\label{fig:#1}
	\end{figure}
}

\newcommand{\figs}[6]{
	\begin{figure}[!ht]
	\centering
	\subfigure[#5]{
		\includegraphics[width=14cm]{img/#2}
		\label{fig:#2}
	}
	\subfigure[#6]{
		\includegraphics[width=14cm]{img/#3}
		\label{fig:#3}
	}
	\caption{#4}
	\label{fig:#1}
	\end{figure}
}
\newcommand{\figsSmall}[6]{
	\begin{figure}[!ht]
	\centering
	\subfigure[#5]{
		% width=14cm
		\includegraphics[width=5cm]{img/#2}
		\label{fig:#2}
	}
	\subfigure[#6]{
		% width=14cm
		\includegraphics[width=5cm]{img/#3}
		\label{fig:#3}
	}
	\caption{#4}
	\label{fig:#1}
	\end{figure}
}


\title{\sc Optimizing 3D models from 2D images}

\author{T. Kostelijk\\mailtjerk@gmail.com}

\begin{document}
\maketitle


\section{Augmenting the 3D building}
\subsection{Introduction}
% write THIS AT LAST
In the previous section we saw the equivalence between the skyline and the building contour, we also saw how this skyline was detected and how every point on this skyline was projected on the buildingwall it most likely presents. 
This 3D pointcloud is an interesting result but it is not enough for our aim, to augment the basic 3D model.
First of all this pointcloud included a lot of noise caused mostly by occluding objects like tree's. How do we detect those outliers?
% why not ransac?
And if we know which data belongs to the building contour we could use the developed projection techniques from section %TODO
to obtain parts of the building contour.
The next question arises is, given this building contour parts, how do we augment the 3D model in a clean way?
We developed an algorithm that adresses these questions.
%before we take these next steps we take one step back and go to the 2d images.
	%we saw that the skyline detector outputs a binary image where a white pixel is considered as highly from the skyline and therefor the building contour.
	%todo
\subsection{The algorithm}
Let us first define the input of the algorithm:
\begin{itemize}
	\item a set of calibrated binary 2D images where white indicates a skyline pixel which has a high proability of being the building contour
	%TODO explain calirated somewhere eerder in het report
	\item a rough 3D model extracted from an open streetview database
	%TODO introduce the 3D model somewhere eerder in report
\end{itemize}
The output will be an updated 3D model, this output is developed in several steps:\\
First some operation are done in 2D, the individual skyline pixels are grouped into line segments using an approach called the Hough transform. In the next step every line is heuristicly associated to (and projected on) a specific wall in 3D. The third step is to combine the projected line segments, which produces and estimate of the height of each buildingwall. In the last step these heights are used to augment the walls of the rough 3D model.
We will now elaborate on each step.

%why fit lines on 2d and not in 3d
%cheaper easyer?
\subsubsection{Extracting line segments, Hough transform} % Houghlines
	%The aim of this step is to use the output of the skyline detector to extract line segments. These line segments can be used to update to indicate the building contour and update the 3D model.

	%chaining my ass of here :)
	% full motivation 
	The main aim of this part is to extract structure from the output of the skyline detector. This includes the removement of outliers (caused by for example an occluding tree) and the extraction of straight line segments. 
	To remove the outliers we put the process upside down: we are detecting the inliers and consider the rest as outliers.\\
	But do we define an inlier? If we take the contour of an average building it is mostly formed by straight lines, assuming the roof is flat.
	% todo example images
	If a bunch of skyline pixels lie on the same line they probably represent a building countour.
	% TODO what if a plain flys by? note in drawbacks of method
	The problem of finding the building contour is now boiled down to finding skyline pixels that lie on a straight line.
	\\
	A famous method for extracting line segments is the the Hough transform (invented by Paul Hough).
	We will explain this method in short, details regarding this method are found in the Appendix.A??%TODO
	In the Hough transform, a main idea is to consider the characteristics of a straight line not as its image points $(x1, y1)$, $(x2, y2)$, etc., but in terms of the parameters of the straight line formula $y = mx + b$. i.e., the slope parameter $m$ and the intercept parameter $b$.
	If a skylinepixel is found the Hough transform increases a vote value for every valid line ($m$,$b$ pair) that crosses this particular point. This is done in a very efficient way, see Appendix A?? %TODO 
	for details.
	The lines ($m$, $b$ pairs) that receive a large amount of votes contain a large amount of skyline pixels. Because it is a straight line it most likely represents a part of the building contour. 
	Optional the Houghtransform returns the start- and endpoint of the lines which is in our case useful. 
	%Because of oclusion the houghlines could return two seperate line segments which is originaly one line which is %onderroken.
	\\
	Results of the Hough transform on the 2D output of the skyline detector are displayed and evaluated in the result section (section %TODO
	).

\subsubsection{Heuristic Line-Wall determination}
	% what do we have what is the input
	From the Hough transform we have a bunch of 2D line segments which represent parts of the building contour. 
	As we agreed in our assumption that every line segment probably presents (a part of) the upperside of a wall of the building.
	It looks like its time to augment these walls but in fact we don't know which linesegments is associated with which wall.
	This section explains how the linesegments are associated with the right wall.\\

	It would be straightforward to use the method of section %TODO
	Instead of of a skyline pixel we could take the linesegments endpoints and project it onto the buildingwalls. The wall with the shortest distance will be assigned to the linesegment.

	%And to update the specific wall we first need to know with a high probability of being correct which wall the line segment presents
	But this method introduces a problem: some of the line segments have endpoints that lie at the corner of the building. These line segments could easily be associated with the neighboring wall. Because the 3D model is a rough estimate this could lead to bad results.
	In the cornercase it is not clear to which wall the linesegment belongs because both endpoints do not agree on the same wall. To solve this problem some heuristic methods are developed and tested. The following heuristic is both simple and effective.
	The heuristic uses the importance of the middlepoint of the linesegment. This middle point has a low change of being on a buildingcorner and the on average biggest change of being on the wall we are looking for.
	Therefor we discard the endpoints and use the middlepoint the endpoints to determine the right wall.
	As in the previous section %TODO
	this middlepoint is intersected with all plaines spanned by the walls. The linesegment is stored to the wall with the shortest distance.
	The output of this part of the algorithm is for every wall a bunch of associated linesegments originated from different views.

	%TODO example image

\subsubsection{Wall height estimation}
	In the previous section we adressed the problem that the endpoints of a line segment may not agree on the wall.  We also introduced a method to associate a line segment with the right wall. In this section this information is used to estimate the height of the wall which is in the next section used to update the 3D model.\\
	Now all line segments are associated with a certain wall we reproject the linesegment from the different views on the wall. The reprojection is done by intersecting both endpoints of the linesegment to the plane that is spanned by associated wall.
	Next the 3D intersection points are collected and averaged, this gives us an average of the midpoints of the projected linesegments. We do this for every wall seperately returning the average height of the line segments.
	These averages are then used as the new heights of the walls of the building.
	%TODO image

\subsubsection{Augmenting the 3D model} %todo other word then augment?
	We made an assumption that a building consists of a flat roof (note that the walls may have different heights but the roof should be flat).
	In the previous section we calculated the new individual heights of the walls the building. 
	This is propagated to the 3D model by adjusting the location of the existing upper cornerpoints of the walls. We copy the bottomleft and -right cornerpoints and add the estimated height from the previous section to its y-value.
	% TODO image
	% TODO footnote:
	FOODNOTE:Note that this method assumes that the walls are alligned on the y-axis, which is in our dataset the case.
	%TODO iets zeggen over het ground plain, zie ook costins werk

	

\subsection{Results (What did I find)}
Lets rehearse the output of the skyline detector.
\fig{outputSkylineIm3-3.eps}{}{0.5}
The following image shows the top 3 longest Houghlines. It is clear that it fits
well on the building contour.

The next image displays the line segments projected onto their associated walls.
For a clear view we only picked three walls of the building.
The red cross in the middle of the line is the average of its endpoints.
\fig{outputHoughlines2d.eps}{}{0.5}

In this image the midpoints of the projected linesegments are averaged. This
point is used as the new height of the building. As can be seen from the
different plane colors, the building walls are changed to their individual height.
\fig{outputHoughlines3d.eps}{}{1.0}

\subsection{Discussion (What do my results mean to me and why)}
???


%TODO cons pros of the flat roof assumption
%TODO vet goed dat ik dus mijn algo different wallheights aan kan, promoten!

% midpoint heuristic
\subsection{Conclusion and Future work}
The flat roof assumption could be stretched to a roof consist of two planes
\\
The middle point of the line is projected to all walls, it could be a significant speedup to reduce the set of walls to only the walls that are found at the endpoints. This is not done because both endpoints could lie on a cornerpoint making it possible that none of the endpoints refer to the correct wall.




% TODO
%TODO what are the mathematics
%TODO write something about complications and wat je er aan hebt
%use power of illustration, lots of example images! ask myself is there stil something left where I can add an illustration??
% read again and check level of detail, is it everywhere the same?
% do I do enough chaining?
% read again on other sujects in mind see how te wr thesis

%TODO spell check!!

\end{document}
