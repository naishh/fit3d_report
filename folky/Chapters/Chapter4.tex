\chapter{Domain} % Write in your own chapter title
\label{Chapter4}
\lhead{Chapter 4. \emph{Domain}} % Write in your own chapter title to set the page header
\section{Guerrilla}
Guerrilla is a game development studio, based in the heart of Amsterdam, the
Netherlands. It was formed at the beginning of 2000 as a result of a merger
between 3 separate Dutch-based developers. The company now employs 130
developers, designers and artists, encompassing 20 different nationalities. The
first game released by Guerrilla, Shellshock: Nam '67 was developed for the PC,
Xbox and PlayStation 2 and published by Eidos Interactive. In 2004, Guerrilla
signed an exclusive deal with Sony Computer Entertainment. Under that deal,
Guerrilla developed games exclusively for Sony's consoles (PlayStation 2,
PlayStation 3 and the PlayStation portable). After the release of Killzone for
the PlayStation 2 (2004), the company was acquired by Sony Computer
Entertainment in 2005. It went on to release Killzone: Liberation for
PlayStation Portable (2006), and KillZone 2 (2009). At the time of writing this
thesis, the company just released a new title for the PlayStation 3 called
Killzone 3.

\section{Killzone 3}
Killzone 3 (KZ3) is a first-person shooter (FPS) game. Figure
\ref{fig:killzone} shows a typical ingame scenario from KZ3 as viewed by the
player.
\begin{figure}[!ht]
\centering
\includegraphics[height=230px]{kz3fps.eps}
\caption{Killzone 3 ingame first-person view}
\label{fig:killzone}
\end{figure}
Like most 3D shooters, KZ3 offers a variety of playing possibilities with
friends or online. It distinguishes three different modes: Singleplayer,
cooperative play and multiplayer. This thesis focuses on the multiplayer mode.

\subsection{Multiplayer}
A Multiplayer online game is a multiplayer video game which is played via a
game server over the Internet, with other human players around the world.
Players either compete against each other (individually or in teams/clans) or
cooperate with each other against a common enemy (e.g. an AI). In contrast to
singleplayer mode, the game is played on one single stage only. Since these
games are not centered around one player, when a player dies, the game is not
restarted. Instead, these games continue and the player that died will have the
opportunity to rejoin the game. \\
A multiplayer game mode is defined by a set of rules and regulations that
specify game objectives, win/lose scenarios and conditions for scoring and
ranking on team and individual basis. For any game mode, points are rewarded
for killing enemies. However, many game modes define multiple different
(primary) objectives and therefore require different strategies to win. The
nature and amount of objectives vary among the different game modes and can
even be different for the opposing teams. For instant in the symmetric game
mode ``Bodycount'', the teams have to kill as many players in the opposing team
as possible, where each kill gains a point. In the game mode ``Assasination'',
a non-symmetric mode, team one has to assassinate a single key player in team
two, which team two has to defend. For this thesis, we will create strategies
for the symmetric game mode ``Capture and Hold''. 

\subsubsection{Capture and Hold}
In this game mode, there are three key areas on the map that can be captured
by either team. To gain control over an area, players must stand within the
capture radius of the area and must make sure that no enemy soldiers are within
capture range of that same area during the capture process. If a team succeeds
in keeping away the enemy long enough, the area will change ownership to the
capturing team. Once an area is captured the team receives points for each
time step the area is under their control.  If the maximum amount of points is
gained (defined by the game) the team wins or, alternatively, if the time limit
is reached the team with the most points wins. When both teams have the same
amount of points the game results in a draw.

%\subsubsection{Bodycount}
%\subsubsection{Assasination}
%\subsubsection{Search and Destroy}
%\subsubsection{Search and Retrieve}
%\subsubsection{Capture and Hold}

\subsection{Multiplayer AI Design Overview}
Singleplayer and coop AI differ greatly from multiplayer AI. The reason for
this difference is that NPCs found in singleplayer games, have a different role
from those in multiplayer games.\\
In order to put the multiplayer bot behavior -- at a strategic level -- in
perspective, this section gives a general overview of the entire AI
architecture \cite{killzone2,killzone2-presentation}.  Figure
\ref{fig:designoverview} shows a simplified overview of the AI hierarchy as
defined in Killzone 3.
\begin{figure}[!ht]
\centering
\includegraphics[height=250px]{DesignOverview.eps}
\caption{Simplified design overview of the AI architecture}
\label{fig:designoverview}
\end{figure}
The top layer shows the decision system at the strategic level, often referred
to as commander or general\footnote{``\emph{Strategy without tactics is the
slowest route to victory. Tactics without strategy is the noise before
defeat.}'' -- Sun Tzu.}. This is the level where the method, proposed in
Chapter \ref{Chapter5}, is focused on.  The subsections below will explain
each layer in more detail (though still at high level). Note that both the
commander and the group are concepts within the AI architecture, they do not
represent actual entities in the virtual world.
\subsubsection{Individual}
The individuals define the actual NPCs. They observe their surroundings through
visual and auditive stimuli modeled after how humans observe the world. This
prevents stupid mistakes such as an NPC ``mysteriously'' knowing that an enemy
is behind him. These observations about the world are stored in a world fact
database. The world fact database is local and different for each individual
and used by the HTN-planner to create behavorial plans using orders.  These
orders include reloading weapon, firing weapon, going in cover, blind fire,
etc.
\subsubsection{Group}
A group, as the name implies, defines a set of individuals. These individuals
form a small military unit often referred to as a squad\footnote{The terms
squad and group are used interchangeable throughout this thesis.}.  The group
is responsible for e.g. defending an area or capturing a strategic point.
Groups are both created and controlled at the commander layer using commands.
Each group also uses a unique world fact database which stores information such
as where each individual is located and which group it belongs to. The squad is
responsible for the coherency of the individuals that belong to it during
movement.
\subsubsection{Commander}
The commander is the top layer in the AI. Its actions are performed on the
strategic level. The commander is responsible for the following tasks:
\begin{itemize}
\item{Squad creation}
\item{NPC to squad allocation}
\item{Commands for squads}
\end{itemize}
The commands that can be sent to squads include: Going to a certain waypoint,
attacking an entity, defending a marker or entity. The squads, in turn, can
report back the status of their progress. These commands form the primitive
actions which, when put in the right sequence, form a sensible plan. We
expanded the HTN architecture to support the commands above for the commander
such that we can encode a strategy in a domain. Given this domain, the
HTN-planner can construct a plan at the start of e.g. a \emph{capture and hold}
game. Figure \ref{fig:ch-example} shows how the command sequence of a squad
capturing an area is build.
\begin{figure}[!ht]
\centering
\begin{verbatim}
(:method (capture_areas ?inp_area_list)
  (branch_areas_captured
    (and (call eq (call get_list_length ?inp_area_list) 0)
         (= ?squad_index (call get_last_created_squad_index))
         (= ?squad (call get_squad ?squad_index))
    )
    (    (!end_command_sequence ?squad)
    )
  )
  
  (branch_capture_area
    (and (= ?area_index (call get_list_item ?inp_area_list 0))
         (= ?area_list (call remove_list_item ?inp_area_list 0))
         (= ?squad_index (call get_last_created_squad))
         (= ?squad (call get_squad ?squad_index))
    )
    (    (!order_squad_custom ?squad capture ?area_index)
         (!order_squad_custom ?squad advance ?area_index auto)
         (capture_areas ?area_list)
    )
  )
)
\end{verbatim}
\caption{Area capture example domain}
\label{fig:ch-example}
\end{figure}
The method requires a list of areas indices as argument and orders a squad to
capture the area and advance. While there are still areas that require
capturing, the method assigns a squad. Finally an ``end of command sequence''
is given.
\\\\
Applying the HTN-planner at this layer separates the domain language from the
problem solver. This separation allows for the application of reinforcement
learning integrated within the HTN-planner. Integrating the RL algorithm on
this level enables machine learning the best strategies across multiple game
modes against various fixed opponents.
\subsubsection{C++ Strategy Implementation}
Currently the strategical logic is implemented in C++. This is done using an
objective based approach. An objective is implemented as a C++ class which
defines the desired amount of bots and squads to be accomplished. Some examples
of objectives are:
\begin{itemize}
\item{Attack Entity}
\item{Defend Marker}
\item{Escort Entity}
\end{itemize}
Every logical update, the objectives and their importances are computed using
hardcoded heuristics and squads get assigned or reassigned to these objectives.
This process continues until the game is over.
