\section{Introduction}
When we humans look at an urban scene we immediately can tell which part
represents a
building, a tree, a door, a window or a parked car.
Even if the scene suffers from high occlusion (a tree occluding the largest part
of a building) or extreme perspective distortion (a building seen from the
corners of your eye) we perform this task with a very high accuracy.
For a computer system however, this task is far from trivial.\\

Let us address the question that arises many times in Artificial Intelligence:
Why are we humans so good in this task? What can we learn from ourselves 
and how can we apply this on a computer system?\\

The most important reason of our excellent visual perception is that we combine 
a series of depth cues (which enables us to experience depth) with top down
processing (which enables us to classify objects).\\

One of the most important depth cue is stereopsis.  We use two eyes and look at
the same scene from slightly different angles.  This makes it possible to
triangulate the distance to an object with a high degree of accuracy. \\

%---
%If an object is far away, the disparity of that image falling on both
%retinas will be small. If the object is close or near, the disparity will be
%large.
Besides the depth cue we use top down processing to perceive objects.  If we want
to perceive a window, we tap into the neurons that are activated according to
our (generalized) description of a window.  I.e. because windows are often
rectangular shaped and the color stands out we tap into the neurons that
perceive straight parallel lines, orthogonal corners and intense color changes.
These visual processes are extremely informative if we want to build a computer system
that acts accordingly.  \\

This thesis is about our work of an automatic system that adds semantics to a
urban scene inspired on the human brain. We focus on the detection of the
building contour, the 3D information of the building and the detection of
windows.  The product is a 3D annotated scene where the building and windows are labeled.\\

Before we explain our methods let us first share the variety of applications 
that use semantical interpretation of urban scenes.  \\

\subsection{Application examples}
\paragraph{3d City models} 
	Manual creation of 3d models is a time consuming and expensive procedure.
	Therefore semantic models are used for semi automatic 3d
	reconstruction/modelling.
	 %[Procedural Modeling of Buildings].  
	The semantic understanding is also used in 3d city models which are
	generated from aerial or satellite imagery.  The (doors and) windows are mapped to the detected 3D model to increase the level of detail. 
	Some other applications can automatically extract a CAD-like model of
	the building surface.

\paragraph{Historical buildings documentation and deformation analysis}
	In some fields of research, historical buildings are documented. The complex
	structures that are contained in the facades are recorded and reconstructed.
	Window detection plays a key role in this. 
	Another field of research is the analysis of building deformation in areas
	containing old buildings.  Window detection provides information about the
	region of interest that could be tracked over time for an accurate
	deformation analysis.
	%[A SEMI-AUTOMATIC IMAGE-BASED MEASUREMENT SYSTEM]

\paragraph{Interactive 3D models}
	There are some virtual training applications that are designed for
	emergency response who require interaction with a 3d model.  
	For the simulation to be realistic it is important to have a model that is
	of high visual quality and has sufficient semantic detail (i.e. contains
	windows).  This is also the case for a fly-through visualization of a street with
	buildings.
	Other applications that require semantic 3d models are virtual tourism,
	visual impact analysis, driving simulation and military simulation systems.
	\fig{p_simulation_people.eps}{Simulation environment}{0.3}

\paragraph{Augmented reality}
	Some mobile platforms apply augmented reality using facade and window
	detection to make an accurate overlay of the building. An example overlay is
	the same building but 200 years earlier.  Semantical information is used to
	not only identify a respective building, but also find his exact location in
	the image.  The accuracy and realistic level of the 3D model are vital for a
	successful simulation.  And because the applications are mobile, very fast
	building understanding algorithms are required.  Window detection plays an
	important role in these processes as the size and location of the windows
	supply an effective descriptor that can be used for robust and fast building
	identification.  Furthermore it provides an accurate alignment of the
	overlay.

\paragraph{Building recognition and urban planning}
	Building recognition is used in the field of urban planning where the
	semantic 3d models are used to provide important references to the city
	scenes from the street level.  Building recognition is done by using large
	image datasets where the buildings are mostly described by local information
	descriptors.  Some approaches try to describe the 3D building with laser
	range data. Some methods fuse the laser data with ground images. However,
	those generated 3D models are a mesh structure which do not make the facade
	structure explicit.  For a more accurate disambiguation, other types of
	contextual information are desired.  The semantical interpretation of the
	facade can provide this need.  In this context, window detection can be used
	as a strong discriminator.

We can conclude that semantic interpretation plays an important role in the
interpretation of urban scenes and is applied in a wide range of domains.  

\subsection{Thesis outline}
The outline of this thesis is as follows:\\ We start with explaining some basic
computer vision techniques in Chapter 2.  These techniques are the driving force
behind the algorithms used in both skyline detection and window detection.  In
Chapter 3 we explain our first method of scene interpretation.  We simulate the
eyes by using images taken from different angles to extract the skyline in a
images.  In Chapter 4 this result is used to extract the contour of a building.
Next we combine this result with projective geometry to extract a 3D model of
the building.  In Chapter 5 we present two window detection methods that are
very similar to the way humans detect windows. We respectively detect the
windows of an unrectified and a rectified scene. The thesis finishes with an overall conclusion.  \\  

Many methods used in this thesis are distinct, therefore we choose to present
the discussion, results and future research for each method separate.  Also the
chapters are independent from each other. Therefore, a reader only interested in
one particular topic may read only the associated chapter describing it while
skipping the other chapters.



