\section{Window detection}
\label{chap:windowDetection}
\subsection{Introduction}
% bruggetje maken, van 3d model een vb applicatie windowdetection
\subsubsection{Related work}

\subsection{Method I: Connected corner approach} 
\subsection{Method II: Histogram based approach} 
\subsubsection{Situation and assumptions}
In this method we assume the wall containing the windows to be rectified.
To be more precise we assume the windows to have orthogonal sides.
Furthermore we assume the windows to be alligned.

The main idea is that we extract the alignment of the windows based on
histogramming Houghlines on projected coordinates.

input 
output


\paragraph{Alignment lines}
%explain pipe line
%(color transform)
%edge extraction
%houghline extraction
histogram based approach, the idea
We introduce the concept alignment line. We define this as a horizontal or
vertical line that alignes multiple windows. In image %TODO the green dotted
green and we represent the alignment lines as two groups, horizontal (red) and
vertical (green) alignment lines. The intersection of both groups give a good
indication of the corners of the windows.

The extraction of the alignment lines consist of several steps.

First we extract the coordinates of the endpoints of the Hough transformed line
segments and store them in two groups, horizontal and vertical (Figure
\ref{fig:houghlineEndpoints}). 

We project the coordinates to the orthogonal direction of its group. This means
that the horizontal Houghlines are projected in the Y direction and the vertical
Houghlines are projected in the X direction, leaving the data in two groups of 1
dimensional coordinates.

We calculate two histograms H(orizontal) and V(ertical), containing respectively $w$
and $h$ bins where $w x h$ is the dimensions of the image.

peak extraction


Images:
edge image
example image of start end coordinates houghlines



\subsection{Method III: Feature detection approach}
	harris corner

\subsection{Fusing the methods}

\subsection{Results}
\subsection{Discussion}  % (What do my results mean to me and why)
\subsection{Conclusion and Future work}




% title: Graph Theory and Mean Shift Segmentation Based Classification of Building Facades
% 
% Most of the previous algorithms are computationally expensive and not suitable for real time applications. Because
% of the requirements of real-time applications robust and fast
% classification of facades is still an open research topic
% 
% 
% Image-based Procedural Modeling of Facades
% 
% 
% 
% refs:
% (canny edge detector:)
% 
% Canny, J., 1986. A computational approach to edge detection.
% IEEE Transactions on pattern analysis and machine intelli-
% gence pp. 679–698.
% 
