% set ignorecase



% todo's



% list of figures
% -bar and barh in fig mergen
% -grayvalues voting
%	- values that fall out of cluster are displayed striped
% -grayvalues voting with clusterlines
% 	- or clustervalues in colored bar graph


\section{Window detection}
\subsection{Updated/New since 28-3-2012}
alle feedback verwerkt\\
explanation different types of connected corners\\
Future research-> Window alignment refinement\\
Discussion -> nummeric evaluation of connected corner result in \\
ik werk niet meer met houghline eindpunten maar met de gehele lijn.\\
ivm onduidelijkheid rijen en pixelrijen gebruik ik nu ipv rectangular areas, en rows, : blocks en blockrows\\
Facade rectification\\
Improved window alignment lines\\

\subsection{Q}
Is it useful to include the connected corner result of the occluding tree?
\ref{fig:w_Dirk5_ImcCorner_cCorner.eps}
Zal ik de presentatie met de blokken met grijswaarden weglaten?

% todo explain efficiency of Houghline coordinate transformation (instead of
% transforming the image
\label{chap:windowDetection}
\subsection{Introduction}
This chapter deals with one of the tasks of semantic urban scene interpretation, Window detection. 
Semantic interpretation of urban scenes is used in a wide range of applications.

\paragraph{3d City models} 
	Manual creation of 3d models is a time consuming and expensive procedure.
	Therefore semantic models are used for semi automatic 3d
	reconstruction/modelling.
	 %[Procedural Modeling of Buildings].  
	The semantic understanding is also used in 3d city models which are
	generated from aerial or satellite imagery.  The detected (doors and)
	windows are mapped to the model to increase the level of detail. 
	Some other applications can automatically extract a CAD-like model of
	the building surface.

\paragraph{Historical buildings documentation and deformation analysis}
	In some field of research, Historical buildings are documented. The complex
	structures that are contained in the facades are recorded and reconstructed.
	Window detection plays a central role in this. 
	Another field of research is the analysis of building deformation in areas
	containing old buildings.  Window detection provides information about the
	region of interest that could be tracked over time for an accurate
	deformation analysis.
	%[A SEMI-AUTOMATIC IMAGE-BASED MEASUREMENT SYSTEM]


\paragraph{Interactive 3d models}
	There are some virtual training applications that are designed for
	emergency response who require interaction with a 3d model.  
	For the simulation to be realistic it is important to have a model that is
	of high visual quality and has sufficient semantic detail (i.e. contains
	windows).  This is also the case for a fly-through visualization of a street with
	buildings.
	Other applications that require semantic 3d models are virtual tourism,
	visual impact analysis, driving simulation and military simulation systems.
	% todo uncomment this
	%\fig{p_simulation_people.eps}{Simulation environment}{0.1}

\paragraph{Augmented reality}
	Some mobile platforms apply augmented reality using facade and window
	detection to make a accurate overlay of the building. An example overlay is the
	same building but 200 years earlier.  Semantical information is used to not
	only identify a respective building, but also find his exact location in the
	image.  The accuracy and realistic level of the 3d model are vital for a
	successful simulation.  And because the applications are mobile, very fast
	building understanding algorithms are required.  Window
	detection plays an important role in these processes.

\paragraph{Building recognition and urban planning}
	Building recognition is used in the field of urban planning where the semantic 3d
	models are used to provide important references to the city scenes from the
	street level.
	Building recognition is done using large image datasets where the
	buildings are mostly described by local information descriptors.  
	Some approaches try to describe the 3D building with laser range data. Some methods fuse the laser data with
	ground images. However those generated 3D models are a mesh structure which doesn't make the facade structure explicit.
	For a more accurate disambiguation, other types of contextual information are
	desired.  The semantical interpretation of the facade can provide this need.
	In this context, window detection can be used as a strong discriminator.\\

%todo write big part about what is new /different / importand in my approach


We can conclude that window detection plays an important role in the
interpretation of urban scenes and is applied in a wide range of domains.  This
chapter presents two developed methods for robust window detection.

We start with discussing related work and putting our work in context.  Then we
describe a window detection approach that is invariant to viewing direction.
After this we present our second method that assumes orthogonal and aligned
windows.  Finally we show and discuss results. 


\subsection{Related work}
A large amount of research is done on semantical interpretation of urban scenes. First we
discuss related work that has a big overlap with our approach in detail.
After this, we briefly discuss the research that is done on window detection
using other approaches.


\subsubsection{Similar approaches}
Pu and Vosselman \cite{Pu_refiningbuilding}
use laser images together with Hough line extraction to reconstruct facade details.  They solve inconsistency between laser and image data and improve the alignment of a 3d model with a matching algorithm.  In one of the matching strategies they compare the edges of a 3d model to Hough lines of ground images.
They match the lines by comparing the angle, location and length difference of the model edges with the extracted Houghlines.  These criteria is also used in our approach.\\
They also detect windows and use them to provide a significant better alignment of the 3d model.
The windows are extracted from the holes from laser points of a wall, these results where far from accurate.\\
To summarize, the work of Pu and Vosselman provides a useful practical application of window detection and it amplifies the need for a robust window detection technique that is independent of laser data.\\


In \cite{Recky_kmeans} Recky et all developed a window detector that is build on
the primary work of Lee and Nevatia \cite{Lee_extraction} (which is discussed
next).
In order to be able to assume aligned windows they rectify the facade.  After this they apply a threshold on an orthogonal projection of the extracted edges. 
For example they use a vertical edge projection to establish the horizontal division of the windows which is very similar to our approach,.\\
%raises a to many floats error
\fig{p_projection_profiles.eps}{Projection profiles of Lee and Nevatias work}{0.1}
The next step, labeling the areas containing windows, is however very different as they use color to disambiguate the windows.
To be more precise, they convert the image to CIE-Lab color space and use k-means to classify the windows.
Although this method is robust, both color transformation and k-means clustering are very computational expensive.
In our method we use the same source, edge information, for the window alignment and for the window labeling.
As we don't require color transformation and only apply math on line segment
endpoints, our algorithm performs in real-time.

As in the work of Recky et all \cite{Recky_kmeans} Lee et all
\cite{Lee_extraction}
perform orthogonal edge projection to find the window alignment.  As different
shape of windows can exist in the same column, they use the window alignment as
a hypothesis.  Then, using this hypothesis, they perform a refinement for each
window independently. More on this in Future research.

%%NEW PAPERS 
%[A model-based method for building reconstruction]
	%brute force matching of window primitives
	%make 3d models and recover the geometry of a building


\subsubsection{Other approaches}
Muller et all \cite{Muller_procedural} detect symmetry in the building. The
symmetry is detected in the vertical (floors) and horizontal (window
rows) direction.
The use shape grammars to divide the building wall in tiles, windows, doors etc.
The results are used to derive a 3d model of high 3d visual quality.
\figsSmall{label_window_nice_geometry}{p_window_nice_geometry1.eps}{p_window_nice_geometry2.eps}{Results of Muller et all}{original texture}{reconstruction}

Using a thermal camera, Sirmacek \cite{Sirmacek_thermal}
detects heat leakage on building walls as an indicator for doors or windows.
Windows are detected with L-shaped features as set of \emph{steerable filters}.
The windows are grouped using \emph{perceptual organization rules}.

Ali et all \cite{Ali_facades}
describe the windows with Haar like features which are fed into a (Ada boost) cascaded decision tree.





% handige note
% \paragraph{Facade classification}
% That shows the importance of the facade classification
% study in three-dimensional city modeling. Burochin et al. [3]
% proposed a segmentation method to detect repetitive structures
% like windows in close-range optical images. For segmentation
% they defined a model by considering shape and reflectance
% Fig. 1. Overview of the proposed approach (Facade1 test image, segmentation
% result with facade graph, and classification result of the algorithm
% respectively.)
% of a window, then they applied matching process to find
% correspondence between model and image. In [2], Ali et
% al. gave summary of the researches on window detection.
% They also proposed a window detection system based on
% cascade classifiers. In a following study, Ali et al. proposed
% a system to detect windows in laser scanner data. The laser
% Therefore, they use these variations to detect windows [1]. Lee
% and Nevatia [8] proposed a robust system to detect windows in
% optical images. They extracted window boundaries searching
% for structures that satisfy regularity and symmetry rules. In
% addition to that, they extract three-dimensional models of
% windows by searching for image features. Teboul et al. [13]
% used shape grammars towards fixed tree representations which
% are able to capture a wide variety of building topologies
% for detailed facade segmentation. They obtained very high
% performance even for buildings which are partially occluded
% or which appears under different illumination conditions.







% todo find good location in thesis for this

% todo IMPORTAND UNCOMMENT THESE
% \fig{w_spil6ImOri.eps}{Original image}{0.6}
% \fig{w_Spil1TransCrop1_ImOri.eps}{Rectified image}{0.45}
% \fig{w_Spil1TransCrop1_ImEdge.eps}{Result edge detection}{0.45}
% \fig{w_Spil1TransCrop1_ImHoughResult.eps}{Houghlines with endpoints}{0.45} 
% 



\subsection{Method I: Connected corner approach} 
\subsubsection{Situation and assumptions}
We introduce the concept \emph{connected corner}, this is a corner that is 
connected to a horizontal and vertical line.  
In this method we search for connected corners based on edge information.
The connected corners give a good indication of the position of the windows, as 
a window consists of a complex structure involving a lot of connected horizontal
and vertical lines. 

In this approach the viewing direction is not required to be frontal.
The windows could be arbitrarily located and they don't need
to be aligned to each other neither to the X and Y axis of the image.

\subsubsection{Method}
\paragraph{Edge detection and Houghline extraction}
Edge detection is done as is described in chapter 
\emph{(not included chapter)} % todo
From the edge image we extract two different groups of Houghlines, horizontal and 
vertical.  We set the $\theta$ bin ranges in the Hough transform that control the
allowed angles of the Houghlines to extract the two groups. The horizontal group
has a range of [-30..0..30] degrees, where 0 presents a horizontal line. The vertical
group has a range of [80..90..100] degrees. These ranges seem
to work well on an empirical basis for all datasets.
The results of two images can be seen in Figure \ref{fig:w_Spil1TransCrop1_ImEdge.eps} and
 \ref{fig:w_Spil1TransCrop1_ImHoughResult.eps}.

\paragraph{Extract connected corners}
\fig{cCornerTypes}{First row: different type of connected corner candidates. Second row: the
result the clean connected corner}{0.5} As windows contain complex structures
the amount of horizontal and vertical houghlines is large at these locations.
A horizontal and vertical line is often connected in a corner of a window.  In
this approach we pair up these horizontal and vertical lines to determine
\emph{connected corners} that indicate a window.

Often a connected corner contains a small gap or an extension wich we tollerate,
these cases are illustrated in Figure \ref{fig:cCornerTypes} in the top row.
A horizontal gap a vertical and horizontal gap and a vertical alongation. The
cleaned up corners are given in the bottom row.  When the horizontal and
vertical lines intersect, the gap distance is $D=0$.  When the lines do not
intersect, the distance between the intersection point and the endpoint of the
lines is measured, this is illustrated as dotted lines in Figure
\ref{fig:cCornerTypes}.  Next, $D$ is compared to a \emph{maximum intersection
distance} threshold $midT$.  And if $D<=midT$, the intersection is close enough
to form a connected corner.\\

After two Houghlines are classified as a connected corner, they are extended or
trimmed, depending on the situation. The results are shown in the second row in
Figure \ref{fig:cCornerTypes}.
In Figure \ref{fig:cCornerTypes}(I)  the horizontal line is extended.  Figure
\ref{fig:cCornerTypes}(II) shows that the vertical line is trimmed.  In Figure
\ref{fig:cCornerTypes}(III) both lines are extended.  At last, Figure
\ref{fig:cCornerTypes}(IV) shows how both lines are trimmed.
%todo nieuw fig maken cCornerTypes: rood is horizontaal ook onderste rij


\paragraph{Extract window areas}
To retrieve the actual windows, each connected corner is mirrored along its
diagonal. The connected corner now contains four sides which form a 
quadrangle window area.
All quadrangles are filled and displayed in Figure
\ref{fig:w_Dirk6_ImcCorner_windowFilled.eps} and
\ref{fig:w_Dirk5_ImcCorner_windowFilled.eps}. This result is discussed in section
\ref{sec:results}.










\subsection{Facade rectification}
\subsubsection{Introduction}
In order to apply our second method of window detection
we need the windows on the facade to be orthogonal and aligned.
Therefore we rectify the facade, which can be achieved in a simple or complex way.\\

The simple rectification method uses point to point correspondences. This 
requires annotation of the cornerpoints of the facade that are mapped with the
corners of a rectangle. This mapping is used to calculate a transformation matrix. 
 The downside of this method is that it isn't very accurate.\\
\emph{Isaac/Frans, do you know how I can explain why?}\\

The second method involves the extraction of a 3D plane of the facade. This 
method is more complex and gives more accurate results. It involves a
comprehensive process and lots of research is done in this area. 
Given the related work, the automity of this module and our focus on the annotational part
we used existing software to apply the window rectification.
% todo ref fit3d
We used I. Estebans \emph{FIT3D toolbox}, which extracts a 3D model from a
series of frames.  One of the planes of the 3D model was used to rectify the
facade, making it ready for robust window detection. 

Using the \emph{FIT3D toolbox} we calculated the motion between a series of frames in order to extract a
pointcloud of matching features. This pointcloud is used to extract a plane.
which is used to rectify the facade by a projection process.  We now explain the steps in more detail.

\subsubsection{Method}
We started by taking a series 6 consecutive consequent (steady zoom, lightning, etc. parameters) images of a scene.
The images are chronologic and have sufficient overlap. 
% todo figure
%The images are taken with the same zoom and lightning parameters
%todo example img: thumbnail filmstrip with 6 floriande images
We calculated the motion of the camera between the frames in a few steps
First we extract about 25k SIFT features of each frame.  Then we use
SIFT descriptors to describe and match the features within the consecutive
frames.  Not all features will overlap or match in the frames therefor RANSAC is used to
robustly remove the outliers.  After this an 8-point algorithm together with a
voting mechanism is used to extract the camera motion.

The frames where matched one by one which returns an estimation of the camera
motion that is not accurate enough.  Therefore a 3-frame match is done which
gives more accurate results.  Unfortunately this result comes with a certain
amount of reprojection error, this error is minimized using a numerical
iterative method called \emph{bundle adjustment}.  The final result is a very
accurate estimation of the camera motion.

The next step is to use this camera motion to obtain a set of 3D points
(corresponding to the matching image features).  This is achieved using linear triangulation
method. 

Next a RANSAC based planefitter is used to accuratly fit a plane through
the 3D points. 
%todo figure

\paragraph{Efficient Projecting} 
Now we extracted the 3D plane of the facade, the next challenge is to use this in order to rectify the facade.\\

It would be straight forward to rectify the full image. However this is
computational very expensive as each pixel needs to be projected. To keep the
computational cost to a minimum we project only the necesarry data. Since we
are using Houghlines we project only the endpoints of the found Houghlines. 
(This is allowed because the projective transformation we apply preserves the
straightness of the lines. Note that this means we apply the edge detection and
Houghline extraction on the unrectified image.)\\

If $h$ is the number of Houghlines, the number of projections is $2h$
When we rectify the full image the number of projection is $w$ x $h$, where $w$,$h$ are the width and height of
the image. To give an indication, for the \emph{Spil} dataset %todo
this means we apply 600 projections in stead of 1572864: a factor of almost 3k faster.\\


The Houghline endpoints are projected to the 3D plane we extracted in the same
way as we explained in chapter \ref{chap:skylinedetection}. To wrapup, we send
rays from the camera center trough the houghline endpoints and calculated the
intersection with the 3D plane.  The result is a 3D point cloud where each
point is labeled to there corresponding Houghline.\\

The next step is to transform the facade (and therefore the Houghlines) are
seen upfront. Instead of trasforming the facade we rotate and translate the camera. 
This means the viewing direction (z-axis) of the camera needs
to be equivalent to the normal of the facade plane. Lets denote the
unit vector spanning the original viewing direction as $z$ and the
unit vector spanning the desired viewing direction/normal of the facade $z'$.
We calculated a rotation matrix from the axis-angle representation.\\

%todo ref
This presentation uses a unit vector $u$ indicating the direction of a directed axis, and an
angle describing the magnitude of the rotation about this axis.
This axis is orthogonal to $z$ and $z'$ and can therefore be
calculated by $u = cross(z,z')$ where cross defines the cross product of
two vectors.
The magnitude of the rotation $\alpha$ is equivalent to the angle $z$ between $z'$ given $u$. 
E.g. if the images are taken almost upfront (the facade is almost rectified) $\alpha$ is low.
$\alpha$ and $u$ are used to create a rotation matrix $R$ which is aplied to the 3D point cloud.
The result is a set of rectified 3D points that are grouped tot their houghlines.

\subsubsection{Results}
For the purpose of representation we also calculated a full rectification
of the image, see Figure \ref{fig:w_Spil1TransCrop1_ImOri.eps}.
%todo fig unrectified with houghlines
%todo fig rectified with houghlines


\subsection{Method II: Histogram based approach} 
\subsubsection{Introduction}
From the previous section we saw that from a series of images, a 3D model of a
building can be extracted. Furthermore we saw that using this 3D model the
scene could be converted to a frontal view of a building, where a building wall
appears orthogonal.  This frontal view enables us to assume orthogonality and
alignment of the windows. 
We exploit this properties to build a robust window detector. First we determine
the alignment of the windows and then we label and group the areas that
contain the windows. 

\subsubsection{Situation and assumptions}
To be more precise in our assumptions, we assume the windows have orthogonal
sides.  Furthermore we assume that the windows are aligned. This means that a
row of windows share the same height and $y$ position. For a column of windows
the width and $x$ position has to be equal.  Note that this doesn't mean that
all windows have the share the same size.

\subsubsection{Method}
The extraction of the windows is done in different steps. First we rectify the 
image making the assumptions are valid. The rectification process is done as
described in the previous section. 

%todo
Then the alignment of the windows is determined, this is based on a histogram 
of the Houghlines'. We use this alignment to divide the
image in window or not window regions.  Finally these regions are classified
and combined which gives us the windows.



\fig{w_Spil1TransCrop1_ImHibaap.eps}{(smoothed) Histograms and window alignment lines}{0.45}
\paragraph{Extract Window alignment}
%explain pipe line
%(color transform)
%edge extraction
%Houghline extraction
We introduce the concept alignment line. We define this as a horizontal or
vertical line that aligns multiple windows. In Figure
\ref{fig:w_Spil1TransCrop1_ImHibaap.eps}
%todo update img
we show the alignment lines as two groups, horizontal (red) and
vertical (green) alignment lines.  The combination of both groups give a grid of
rectangles that we classify as window or non-window areas.\\

% MOTIVATION
How do we determine this alignment lines? We make use of the fact that among a
horizontal alignment line a lot of horizontal Houghlines are present, see
Figure \ref{fig:w_Spil1TransCrop1_ImHoughResult.eps}. For the vertical alignment lines
the number of vertical Houghline is high, see green lines in Figure
\ref{fig:w_Spil1TransCrop1_ImHoughResult.eps}.\\

We begin by extracting the pixelcoordinates of Hough transformed line
segments. We store them in two groups, horizontal and vertical.% (crosses in Figure \ref{fig:w_Spil1TransCrop1_ImHoughResult.eps}). 
We discard the dimension that is least informative by projecting the coordinates to
the axis that is orthogonal to its group. 
This means that for each horizontal Houghline the coordinates on the line are projected to the X
axis and for each vertical Houghline the coordinates are projected to the Y
axis. We have now transformed the data in two groups of 1 dimensional
coordinates which represent the projected position of the Houghlines.\\

Next we calculate two histograms H(orizontal) and V(ertical), containing respectively
$w$ and $h$ bins where $w x h$ is the dimension of the image.  The histograms
are presented as small yellow bars in Figure \ref{fig:w_Spil1TransCrop1_ImHibaap.eps}.

The peaks are located at the positions where an increased number of Houghlines
start or end.  These are the interesting positions as they are highly correlated
to the alignment lines of the windows. 

It is easy to see that the number of peaks is far more then the desired number of alignment lines.
Therefore we smooth the values using a moving average filter.
The result, red lines in Figure \ref{fig:w_Spil1TransCrop1_ImHibaap.eps}
, is a smooth \emph{projection profile} which contains the right number of peaks. The peaks
are located at the average positions of the window edges. Next step is to
calculate the peak areas and after this the peak positions. 

Before we find the peak positions we extract the peak \emph{areas} by thresholding the
function. To make the threshold invariant to the values, we set the threshold to 0.5 $\cdot$ max Peak. 
(This value works for most datasets but is a paramater that can be changed).
%The two thresholds are presented as black dotted lines in Figure \ref{fig:w_Spil1TransCrop1_ImHibaap.eps}.\\
Next we create a binary list of peaks P, P returns 1 for positions that are contained in
a peak, i.e. are above the threshold, and 0 otherwise.
% todo latex, afkijken locate my plate
%P(x)  { 1, H(x)>t
%	  { 0, H(x)<=t
We detect the peak areas by searching for the positions where P = 1
(where the function passes the threshold line). 
If we loop through the values of P we detect a peak-start on position $s$ if ${P(s-1),P(s)}={0,1}$
and a peak-end on $e$ if ${P(e-1),P(e)}={1,0}$. 
I.e. if P = 0011000011100, then two peaks are present. The first peak covers positions $(3,4)$, 
the second peak covers $(9,10,11)$.\\

Having segmented the peak areas, the next step is to extract the peak positions. 
Each peak area has only one peak and, since we used an average smoothing filter, the shape of 
the peaks are often concave. Therefore we extract the peaks by locating the max of each peak area. 
These locations are used to draw the window alignment lines, they can be seen
% todo check this
as dotted red lines and dotted green lines in Figure \ref{fig:w_Spil1TransCrop1_ImHibaap.eps}.


\paragraph{Improved window alignment lines}
As you can see in Figure
a few window alignment lines are not found.

% todo occlusion uitgebreid uitleggen, door occlusie stenen wand is venster van raam niet zichtbaar en gaat brick over in weerspiegeling raam wat ongeveer zelfde kleur heeft

The right side of the window frame of the first 4 windows is occluded by the
wall.  The reflection of the window has about the same color as the bricks of
the wall, this means that if the window frame is missing the edge detector
doesn't find a strong edge.
This can be seen in the edge image, and at the low height of the peaks (%todo color
) at these positions.

This means we have to find another way to detect this window alignment.
For the vertical alignment lines we only took vertical lines into account.

Lets examine the projection profile of the \emph{horizontal} Houghlines
projected on the X axis, $X_h$.
On the positions of the desired vertical aligmnent lines there appears to be a 
big decrease or increase of $X_h$ at the window frame. This is because on these
positions a window containing (a large amount of horizontal lines) starts or end.

We detect these big decrease or increases by creating a new pseudo peak profile
$\rho_x$ that takes the absolute of the derivative of $X_h$.
\[\rho_x = abs( X_{h}')\]
%--------------------------------------------------------------------------------------------------------------



%todo
%this is new and so far not done in previous research

% todo in related work zeggen wat ik anders doe
% todo zoomed image of projection profile graphs








The image is now divided in a new grid of blocks based on these
alignment lines. The next challenge is to classify the blocks as window and
non-window areas: the window classification.


\paragraph{Window classification}
Instead of classifying each block independently, we classify full rows and
columns of blocks as window or non-window areas.  This approach results in more accurate
classification as it combines a full blockrow and blockcolumn as evidence for a singular
window. 

The method exploits the fact that the windows are assumed to be
aligned.
A blockrow that contains windows will have a high amount of vertical
Houghlines, Figure \ref{fig:w_Spil1TransCrop1_ImHoughResult.eps}
(green). For the blockcolumns the number of horizontal Houghlines
 (red) is high at window areas.  We use this property to classify 
 the blockrows/blockcolumns. 

For each blockrow the overlap of all vertical Houghlines are summed up.
(Remark that with this method we take both the length of the Houghlines and
amount of Houghlines implicitly into account.)

To prevent the effect that the size of the blockrow influences the outcome, this total value
is normalized by the size of the blockrow.
\[\forall Ri\in \{1..numRows\} : R_i = \frac{HoughlinePxCount}{R_i^{width} \cdot R_i^{height}}\]

Leaving us with $||R||$ (number of blockrows) scalar values that give a rank of a blockrow begin a window area or not.
This is also done for each blockcolumn (using the normalized horizontal amount of
Houghlines pixels) which leaves us with $C$.

\fig{w_Spil1TransCrop1_ImClassRectBarh.eps}{Normalized vertical Houghline pixel count of
the blockrows (R)}{0.6}
If we examine the distribution of $R$ and $C$, we see two clusters appear: one with
high values (the blockrows/blockcolumns that contain windows) and one with low values (non window
blockrows/blockcolumns). For a specific example we displayed the values of $R$ in Figure \ref{fig:w_Spil1TransCrop1_ImClassRectBarh.eps}.
Its easy to see that the high values, blockrow 4,5,7,8,10 and 11, correspond to the
six window blockrows in Figure \ref{fig:w_Spil1TransCrop1_ImClassRect.eps}.

How do we determine which value is classified as high?  A straight forward
approach would be to apply a threshold, for example 0.5 would work fine.
However, as the variation of the values depend on (unknown) properties like the
number of windows, window types etc., the threshold maybe classify insufficient
in another scene.  Hence working with the threshold wouldn't be robust. 

Instead we use the fact that a blockrow is either filled with windows or not, hence
there should always be two clusters.  We use \emph{$k$-means} clustering (with
$k=2$) as the classification procedure.
%todo ref
This results in a set of Rows and Columns that are classified as window an
non-window areas.

The next step is to determine the actual windows $W$.
A rectangular area $w\in W$ that is crossed by $R_j$ and $C_k$ is classified as a
window iff \emph{$k$-means} classified both $R_j$ and $C_k$ as window areas. These are displayed in 
 Figure \ref{fig:w_Spil1TransCrop1_ImClassRect.eps} as green rectangles.

The last step is to group a set of windows that belong to each other. This is done by 
grouping adjacent positively classified rectangles. These are displayed as red
rectangles in Figure \ref{fig:w_Spil1TransCrop1_ImClassRect.eps}.

% todo binary presentatie weglaten?
% As the figure gives a binary representation of the windows it is not possible
% to see the probabilities behind the classification.
% Therefore we developed a kind of probabilistic function. 
% \[P(R_i) = \frac{R_i}{max(R)}\]
% \[P(C_i) = \frac{C_i}{max(C)}\]
% \[P(w) = \frac{P(R_i) + P(C_i)}{2}\]
% As you can see $P$ is normalized, this is to ensure the value of the maximum
% probability is exactly 1. The results can now be relatively interpreted, e.g. if the rectangle's $P=0.5$
% then the system nows for 50 percent sure it is a window, compared to its best window ($P=1$). 
% And, as the normalization implies this, there are always one or more window with $P=1$. 

To get insight about the probabilities that lie behind the individual blockrows and blockcolumns
we designed another representation in Figure \ref{fig:w_Spil1TransCrop1_ImClassRectGrayscaleProb.eps}
The whiter the area the more probable a rectangle is classified as a window.



\subsection{Results}
% todo discuss drain pipe
% todo occlusion uitgebreid uitleggen, door occlusie stenen wand is venster van raam niet zichtbaar en gaat brick over in weerspiegeling raam wat ongeveer zelfde kleur heeft
\label{sec:results}.
We tested both methods on different datasets.

\newpage
\fig{w_Spil1TransCrop1_ImClassRectGrayscaleProb.eps}{Window classification probabilities, white means high.}{0.45}

\fig{w_Dirk6_ImEdge.eps}{Edge detection}{0.6}
\fig{w_Dirk6_ImHoughResult.eps}{Result of $\theta$ constrained Hough transform}{0.6}
\fig{w_Dirk6_ImcCorner_cCorner.eps}{Found connected corners}{0.6}
\fig{w_Dirk6_ImcCorner_windowFilled.eps}{Window regions}{0.6}

\fig{w_Dirk5_ImEdge.eps}{Edge detection (with occluding tree)}{0.6}
\fig{w_Dirk5_ImHoughResult.eps}{Result of $\theta$ constrained Hough transform (with occluding tree)}{0.6}
\fig{w_Dirk5_ImcCorner_cCorner.eps}{Found connected corners (with occluding tree)}{0.6}
\fig{w_Dirk5_ImcCorner_windowFilled.eps}{Window regions (with occluding tree)}{0.6}


% old dataset:
%\fig{w_spil6cCornerImEdge.eps}{Edge detection}{0.6}
%\fig{w_spil6cCornerImHoughResult.eps}{Result of $\theta$ constrained Hough transform}{0.6}
%\fig{w_spil6cCornercCorner.eps}{Found connected corners}{0.6}
%\fig{w_spil6cCornerWindows.eps}{Connected corner as windows}{0.6}
%\fig{cCornerSpilTrans1.eps}{Found connected corners on the rectified image}{0.45}

%todo caption onderstaand fig
\fig{w_Spil1TransCrop1_ImClassRect.eps}{Classified rectangles}{0.45}

\fig{w_SpilFrontal6345_crop1_ImOri.eps}{Original Image}{0.6}
\fig{w_SpilFrontal6345_crop1_ImHibaap.eps}{Window alignment lines and histograms}{0.6}
\fig{w_SpilFrontal6345_crop1_ImClassRect.eps}{Classified rectangles}{0.6}



\emph{todo include images other datasets} \\
\emph{todo compair methods and explain differences}


\subsection{Discussion}  % (What do my results mean to me and why)
\emph{ todo discuss results }

\paragraph{Method I: Connected corner approach} 
Figure \ref{fig:w_Dirk6_ImcCorner_windowFilled} contains 110 windows of which
are 109 detected, this is 99\%. Furthermore there are some False Positive areas,
this is about 3 \%.  The window on the right top isn't detected, this is because
he is smaller then our minimum window width.


% todo ref
The big advantage of this method is that it doesn't require the windows to be aligned.
Furthermore it's robust to a variation in window sizes and types. This makes
this approach suitable for a wide range of window scenes where no or few prior
information about the windows is known.
\emph{todo}


\paragraph{Method II: Histogram based approach} 
%todo
%\% of the windows are found
%\% of the windows are wrongly classified
%\% of the windows are grouped right
%etc



% todo also works on multiple window types
It could be a drawback that the outcome is non-deterministic, as it depends on to the
random initialization of the cluster centers. Our results could be correct by
coincidence.  To exclude this artefact, we ran the cluster algorithm 10 times,
fortunately it resulted in the same classification.

% todo test this on different runs and say something about sensitivity

\emph{todo}\\

\paragraph{Occlusion}
\label{lab:occlusion}
If the image isn't the frontal view of the buildingwall we project the image 
see section ?%todo section rectification
This projection comes with some difficulties, occlusion.  In a few cases an
buiding wall extension (middle of figure \ref{fig:w_spil6ImOri.eps}) a drainpipe
or the building wall itself is occluding a part of the window.  The less frontal
the view, the more occlusion negatively effects the cleanness of the projection.
However, this occlusion artefact is in most cases no problem as the system
combines the windows probabilities.  

\subsection{Conclusion}
% todo nummeriek concluderen
%We showed that projecting the image to a frontal view is a good preprocessing
%step of a robust window detector.
\emph{todo}
%todo

\subsection{Future research}
\subsubsection{Method I: Connected corner approach} 
It would be nice to group the connected corner to groups of subwindows.
The big window that contains subwindows could be found by calculating the convex hull of the red areas in 
Figure \ref{fig:w_Dirk6_ImcCorner_windowFilled.eps}.
The subwindows could be found using a clustering algorithm that groups the connected corners to
a window. For this method it would be useful to assume the window size as this
correlates directly to the inter-cluster distance.
It would also be nice to incorporate not only the center of the connected corner
as a parameter of the cluster space but also the length and position of the of
the connected corners' horizontal and vertical line parts.  The inter cluster
distance and the number of grouped connected corner could form a good source for
the probability of the subwindow.\\

We only developed L-shaped connected corners, it would be nice to connect more
parts of the window to form U shaped connected corners or even complete rectangles.\\
The later is difficult because the edges are often incomplete due to for example occlusion 
or the angle of viewing.


\subsubsection{Method II: Histogram based approach} 
It would be nice to investigate the effect of the occlusion and to examine the
robustness of the window detector under extreme viewing angles.
For example the viewing angle could be plotted against the percentage of
correct detected windows.
% todo  more..


\paragraph{Window alignment refinement}
To get more accurate result or to handle scenes with poor window alligment a refinement procedure could be applied.
As mentioned in the related work, Lee et all \cite{Lee_extraction} applied window refinement.
Although this comes with accurate results, the iterative refinement is a
computational expensive procedure. 
It would be nice to have a dynamic system that is aware of this 
accuracy and computational time trade of. A system that only refines the results when the recources are available.
For example if a car is driving and uses window detection for building recognition the refinement is disabled.
But if the car is lowering speed the refinement procedure could be activated.
Resulting in accurate building recognition which opens the door for augmented reality.

\paragraph{Feature fusion}
Both window refinement and window alignment steps could use some additional
evidence which could be provided by feature based methods.  For example a
\emph{multiscale Harris corner detector} could help an accurate alignment or
refinement of the windows.

