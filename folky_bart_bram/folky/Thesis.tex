%% ----------------------------------------------------------------
%% Thesis.tex -- MAIN FILE (the one that you compile with LaTeX)
%% ---------------------------------------------------------------- 

%% TODO TODO TODO TODO TODO TODO TODO TODO TODO TODO TODO TODO TODO
%% Chapter 2:
%%
%% Chapter 3:
%%   - State for each related work the pitfalls and relate it to my work
%%   - Introduce more references via CIG (found 5 more papers)
%%
%% Chapter 4:
%%
%% Chapter 5:
%%
%% Chapter 6:
%%   - Elaborate on the results in a more descriptive manner
%%      + graphs on comparison
%%      + create graph for hgh on map 2
%%
%% Chapter 7:
%%   - More explanation on the results
%% TODO TODO TODO TODO TODO TODO TODO TODO TODO TODO TODO TODO TODO

% Set up the document
\documentclass[a4paper, 11pt, oneside]{Thesis}  % Use the "Thesis" style, based on the ECS Thesis style by Steve Gunn
\graphicspath{{Figures/}}  % Location of the graphics files (set up for graphics to be in PDF format)

% Include any extra LaTeX packages required
\usepackage[square, numbers, comma, sort&compress]{natbib}  % Use the "Natbib" style for the references in the Bibliography
\usepackage{verbatim}  % Needed for the "comment" environment to make LaTeX comments
\usepackage{algorithm, algorithmic}
\usepackage{amsmath, amsthm, amssymb}
\hypersetup{urlcolor=blue, colorlinks=true}  % Colors hyperlinks in blue, but this can be distracting if there are many links.

%% ----------------------------------------------------------------
\begin{document}
\frontmatter	  % Begin Roman style (i, ii, iii, iv...) page numbering

% Set up the Title Page
\title  {Machine learning strategic game play for a first-person shooter video game}
\authors  {\texorpdfstring
            {\href{folkerthuizinga@gmail.com}{F. Huizinga}}
            {F. Huizinga}
            }
\addresses  {\groupname\\\deptname\\\univname}  % Do not change this here, instead these must be set in the "Thesis.cls" file, please look through it instead
\date       {\today}
\subject    {Learning in games}
\keywords   {reinforcement learning, hierarchical task networks, planning, killzone, guerrilla}

\maketitle

%% ----------------------------------------------------------------
\null\vfill
\pagestyle{empty}  % No headers or footers for the following pages
\newpage
%% ----------------------------------------------------------------
\setstretch{1.3}  % It is better to have smaller font and larger line spacing than the other way round

% Define the page headers using the FancyHdr package and set up for one-sided printing
\fancyhead{}  % Clears all page headers and footers
\rhead{\thepage}  % Sets the right side header to show the page number
\lhead{}  % Clears the left side page header

\pagestyle{fancy}  % Finally, use the "fancy" page style to implement the FancyHdr headers
%% ----------------------------------------------------------------
% The "Funny Quote Page"
\pagestyle{empty}  % No headers or footers for the following pages

\null\vfill
% Now comes the "Funny Quote", written in italics
\textit{``The question of whether a computer can think is no more interesting
than the question of whether a submarine can swim.''}

\begin{flushright}
Dijkstra
\end{flushright}

\vfill\vfill\vfill\vfill\vfill\vfill\null
\clearpage  % Funny Quote page ended, start a new page
%% ----------------------------------------------------------------

% The Abstract Page
\addtotoc{Abstract}  % Add the "Abstract" page entry to the Contents
\abstract{
\addtocontents{toc}{\vspace{1em}}  % Add a gap in the Contents, for aesthetics
%% In the standard field of Artificial Intelligence (AI), the main goal is to find
%% an optimal solution for a given problem, which often requires a form of
%% adaptation or learning. This in contrast to AI in (commercial) video games,
%% where the goal is to provide a rich gaming experience for the end user.
%% Providing this experience requires a careful balance between control by the
%% developers and autonomous, learning AI.
The main goal of Artificial Intelligence (AI) in commercial video games is to
provide a rich gaming experience for the end user. Providing this experience
requires a careful balance between encoding prior knowledge, control by
developers and autonomous, learning AI. 

This thesis proposes a method that maintains a balance between these three
objectives. The method proposed is a novel combination of Hierarchical Task
Network (HTN) planning and Reinforcement Learning (RL). The HTN provides the
structure to encode a variety of strategies by human developers, whereas RL
allows for learning the best strategy against a fixed opponent. The method is
empirically analyzed by applying it on the strategic level of \emph{Killzone 3}
-- a first-person shooter video game. 

From the results we can conclude that the best strategy with respect to a fixed
opponent indeed is learned and when a different opponent is introduced, the
system has the ability to adapt its strategy accordingly. 
}

\clearpage  % Abstract ended, start a new page
%% ----------------------------------------------------------------

\setstretch{1.3}  % Reset the line-spacing to 1.3 for body text (if it has changed)

% The Acknowledgments page, for thanking everyone
\acknowledgments{
\addtocontents{toc}{\vspace{1em}}  % Add a gap in the Contents, for aesthetics
This thesis is the final product of my time as intern researcher at Guerrilla.
I would like to thank my supervisors Shimon Whiteson (Assistant professor at
the University of Amsterdam) and Remco Straatman (Lead AI developer at
Guerrilla). Furthermore I would like to thank all members of the AI team at
Guerrilla for their technical support and constructive feedback and last but
not least my family for their loving support.
}
\clearpage  % End of the Acknowledgments
%% ----------------------------------------------------------------

\pagestyle{fancy}  %The page style headers have been "empty" all this time, now use the "fancy" headers as defined before to bring them back


%% ----------------------------------------------------------------
\lhead{\emph{Contents}}  % Set the left side page header to "Contents"
\tableofcontents  % Write out the Table of Contents

%% ----------------------------------------------------------------
\lhead{\emph{List of Figures}}  % Set the left side page header to "List if Figures"
\listoffigures  % Write out the List of Figures

%% ----------------------------------------------------------------
%\lhead{\emph{List of Tables}}  % Set the left side page header to "List of Tables"
%\listoftables  % Write out the List of Tables

%% ----------------------------------------------------------------
\mainmatter	  % Begin normal, numeric (1,2,3...) page numbering
\pagestyle{fancy}  % Return the page headers back to the "fancy" style

% Include the chapters of the thesis, as separate files
% Just uncomment the lines as you write the chapters

% Chapter 1
\chapter{Introduction} % Write in your own chapter title
\label{Chapter1}
\lhead{Chapter 1. \emph{Introduction}} % Write in your own chapter title to set the page header
The main goal of Artificial Intelligence (AI) in commercial video games is to
provide a rich gaming experience for the end user. Providing this experience
requires a careful balance between encoding prior knowledge, control by
developers and autonomous, learning AI. 

\section{Objective}
This thesis proposes a method that maintains a balance between encoding prior
knowledge, control by developers and autonomous, learning AI. The method
proposed is a novel combination of Hierarchical Task Network (HTN) planning and
Reinforcement Learning (RL). The HTN provides the structure to encode a variety
of strategies, whereas RL allows for learning the best strategy against a fixed
opponent. The method is empirically analyzed by applying it on the strategic
level of the video game called \emph{Killzone 3} (KZ3). KZ3 is a first-person
shooter (FPS), a genre of computer games where the player's viewing perspective
is shown through the eyes of their virtual character.

\section{Contributions}
In this thesis, we try to answer the following questions:
\begin{enumerate}
\item{Can we encode good strategies into a hierarchical task network?}
\item{Is it possible to learn the best strategy with respect to a fixed opponent?}
\item{Is it possible to adapt to a different fixed opponent?} 
\end{enumerate}
The first allows for a clear separation of domain language and problem solver
aswell as easy redesign of certain handcrafted strategies by the developer. The
second and third enable a richer, more personal gaming experience. By building
a learning mechanism on top of the HTN-Planner, the best (w.r.t. the current
fixed opponent) handcrafted strategy can be learned while keeping control over
the individual strategies, see chapter \ref{Chapter5} for the method proposed.

\section{Environment}
The environment in which the proposed method is implemented and tested is a
commercial first-person shooter called Killzone 3 developed by
\emph{Guerrilla}. KZ3 provides a good, real world test bed as its regarded as a
typical, high quality FPS game throughout the world. See chapter
\ref{Chapter4} for a detailed description.

\section{Scope}
In a typical FPS game there are several game modes, each with different
objectives that a team has to achieve to win the game. For the scope of this
thesis, the implementation is applied to the game mode called \emph{Capture and
Hold}.  However, it should be noted that the proposed method itself can be
applied to all existing game modes.\\
The objective of \emph{Capture and Hold} is to gain control over the key areas
on the map that can be captured by either team. To gain control over an area,
players must stand within the capture radius of the area and make sure that no
enemy soldiers are within capture range of that same area during the capture
process. If a team succeeds in keeping away the enemy long enough, the area
will change ownership to the capturing team. Once an area is captured the team
receives points for each time step the area is under their control. If the
maximum amount of points is gained (defined by the game) the team wins or,
alternatively, if the time limit is reached the team with the most points wins.
When both teams have the same amount of points the game results in a draw.

\section{Overview}
Chapter \ref{Chapter2} explains the background required for the proposed
method, chapter \ref{Chapter3} discusses related work from both the gaming
industry and academic world. In chapter \ref{Chapter4} the domain (Killzone 3)
is explained in detail and chapter \ref{Chapter5} introduces the proposed
method.  Experiments and results are presented in chapter \ref{Chapter6} and
finally the conclusions, discussion and future work are given in chapter
\ref{Chapter7}.
 % Introduction

\chapter{Background}
\label{Chapter2}
\lhead{Chapter 2. \emph{Background}}
This chapter describes the theory of two well known fields within AI called
Reinforcement Learning and Hierarchical Task Network Planning onto which the
proposed method is based.

\section{Reinforcement Learning}
Reinforcement learning is a sub-area of machine learning concerned with how an
agent ought to take actions in an environment so as to maximize its long term
reward signal. This agent learns the mapping of situations to actions in order
to maximize an expected reward~\cite{sutton}.  Instead of learning by example,
as is the case with supervised learning, learning desired behavior occurs using
past knowledge (experience) from the environment. The current knowledge of an
agent about its environment is often referred to as a state. When a state
represents the environment such that it is \emph{sufficient statistic} for the
history, it satisfies the so called Markov property.

\subsection{Markov Decision Process}
A Markov Decision Process (MDP) is a reinforcement learning task which
satisfies the Markov property. When the number of states in an environment of
an agent are finite the MDP is a finite MDP. Formally the finite MDP is a tuple
$(S,A,P,R)$, where:
\begin{itemize}
\item{$S$ is the finite state space and $s \in S$ is a state the agent can
observe.}
\item{$A$ is the action space and $a \in A$ is an action that can be performed
by the agent.}
\item{$P$ holds the transition function. When an agent performs action $a$, the
environment makes a transition $s \rightarrow s'$, between the current state
$s$ and its successor state $s'$ with a probability $P(s'|s,a)$.}
\item{$R$ is the reward function defined as $R:S \times A \mapsto \mathbb{R}$,
which returns an expected reward $r$ each time an action $a$ is performed from
$s$. }
\end{itemize}
A policy $\pi$ defines the learning agent's way of behaving at a given time (a
mapping between states and actions) and is formally defined as $\pi : S \mapsto
A$, so when an agent is in a certain state $s$ it can perform action $a =
\pi(s)$. The objective of an agent is to maximize the \emph{return}, i.e. the
cumulative (discounted) reward.

\subsection{Value Functions}
A value function describes the expected discounted return of an agent to be in a certain
state. It is defined with respect to a certain policy $\pi$. A state-value
function has a value $v$ for every state, so for $s \in S$ there exists a value
$v = V^\pi(s)$. Formally the state-value function is defined as:
\begin{eqnarray}
V^\pi(s) & = & \sum_{s'} P(s'|s,\pi(s)) ~ \Big[R(s'|s,\pi(s)) + \gamma V^\pi(s')\Big],
\end{eqnarray}
where $s,s' \in S$ and $\gamma$ is the discount value~\cite{sutton}. This has
been proven by Bellman (1957) and is therefore called the Bellman equation. The
value function $V$ is optimal, denoted as $V^*$, when it maximizes the expected
return in all states $s$.
\begin{eqnarray}
V^*(s) & = & max_a \sum_{s'} P(s'|s,a) ~ \Big[R(s'|s,a) + \gamma V^*(s')\Big].
\end{eqnarray}
An optimal policy is denoted as $\pi^*$ and it is always greedy with respect to
$V^*$, $max_\pi V^\pi(s)$.  Similarly, there exists an action-value function.
This determines the value of taking action $a$ in state $s$ and thereafter
following policy $\pi$, which is formally defined as:
\begin{eqnarray}
Q^\pi(s,a) & = & \sum_{s'} P(s'|s,a) ~ \Big[R(s'|s,a) + \gamma Q^\pi(s',\pi(s'))\Big].
\end{eqnarray}
The optimal action-value function $Q^*(s,a)$ satisfies the equation:
\begin{eqnarray}
Q^*(s,a) & = & \sum_{s'} P(s'|s,a) ~ \Big[R(s'|s,a) + \gamma max_{a'}~Q^*(s',a')\Big].
\end{eqnarray}
Note that there can potentially exist many optimal policies.

\subsection{Reinforcement Learning Algorithms}
The algorithms used in reinforcement learning try to construct an optimal
policy for an unknown MDP. They start with an initial state $s_0$ and on each
time step $t$ perform an action $a \in A$ on $s$ that is executable in $s$. For
episodic tasks, this will change $s$ to $s'$ and a reward $r$ is received until
$s'$ is terminal, in which case the $s'$ is reset to a (new) $s_0$. For
continuing tasks $s'$ is never terminal, meaning the episode will continue
indefinitely.
\\ \\
In Q-learning, a well known algorithm \cite{sutton}, when $s$ is observed, $a$
is chosen, $r$ is received and $s'$ observed an update is performed:
\begin{equation}
Q_t(s,a) = (1-\alpha_t)Q_{t-1}(s,a) + \alpha_t\Big[r + \gamma max_{a'}~Q_{t-1}(s',a')\Big],
\label{eq:qlearn}
\end{equation}
Where $\alpha_t$ is the learning parameter, that typically decreases with time
$t$, which determines how much influence the update has on the current value.
As $Q_t(s,a)$ reaches optimal values, the learning rate should decrease to 0.

\subsection{Action Selection Algorithms}
The simplest action selection rule is to select the action with the highest
expected value. This rule will always \emph{exploit} current knowledge and
never \emph{explore} actions with a lower immediate reward which may end up
being better. A simple alternative that does explore is called
$\epsilon-$Greedy. This method, in a greedy manner, exploits the maximum
expected value with a probability of $1-\epsilon$ and explores with a
probability of $\epsilon$ by selecting a random action.

\subsection{Contextual Bandit Problem}
A standard textbook problem in reinforcement learning is the $n$-Armed Bandit
Problem.  The $n$-Armed Bandit Problem works on a round-by-round basis.  On
each round:
\begin{enumerate}
\item{A policy chooses arm $a$ from $1,\ldots,n$.}
\item{The world reveals reward $r_a$ of the chosen arm.}
\end{enumerate}
As information is accumulated over multiple rounds (or episodes), a good policy
might converge on a good choice of arm. So each episode a single action is
chosen and the reward $r_a$ is observed. In the Contextual Bandit Problem also
a single action is chosen per episode, but the action may differ depending on
the context or the environment. Indeed the Contextual Bandit Problem includes
context in the form of a state.\footnote{An example of this type of problem is
targeted advertisement e.g. the selection of ads alongside an email. The state
$s$ holds a vector of keywords from the mail, the action $a$ is one of many ads
and the reward $r$ wether the ad is clicked on.}

\section{Hierarchical Task Networks}
Whereas reinforcement learning provides a framework to compute an optimal
policy, the action set can be defined using a Hierarchical Task Network (HTN).
An HTN distinguishes two types of tasks: \emph{primitive} tasks or actions that
can be executed directly, and \emph{compound} tasks which are composed of a set of
simpler tasks. Instead of a single action, as is the case in conventional RL, a
decomposed HTN plan consists of a specific sequence of primitive actions. These
specific sequences are encoded within the structure of the HTN itself and
reduce the action space compared to conventional RL.

\subsection{HTN-planner}
The objective of an HTN-planner \cite{shop, shop2} is to produce a sequence of
actions that can solve a particular planning problem. HTN planning occurs by
recursively decomposing compound tasks into smaller tasks. Eventually this
process results in a sequence of primitive tasks (a plan), which need no
further decomposition and can be executed directly. To construct a plan, the
planner takes a planning problem and a planning domain. A planning problem is
formulated as a compound task that is to be decomposed combined with a set of
facts that describe the current world state. Planning domains describe the
problem in a hierarchy composed of compound and primitive tasks.

\subsection{Example}
Figure \ref{fig:htnexample} shows a travel example domain and figure
\ref{fig:htnstate} shows an example planning state. Suppose our planning
problem is \verb|((travel-to uptown))|. The HTN-planner will begin calling the
compound task \verb|travel-to| with \verb|uptown| as argument. Next
the preconditions of the first branch are evaluated. In this case, as \verb|(>= 30 60)| 
evaluates to false, the walking branch is chosen. This results in a
decomposed plan consisting of a single primitive task: \verb|((!walk downtown uptown))|, a long walk indeed.
\begin{figure}[!ht]
\centering
\begin{verbatim}
(:method (travel-to ?y)                         // method name
    (branch_taxi                                // branch name
        (and (at ?x)                            // preconditions
             (distance ?x ?y ?d)
             (have-cash ?cash)
             (= ?fare (* ?d 1.5))
             (eval (>= ?cash ?fare))
        )
        (    (!hail-taxi ?x)                    // primitive task
             (drive-taxi ?x ?y)                 // compound task
             (@set-cash ?cash (- ?cash ?fare))  // operator
        )
    )

    (branch_walking                             // branch name
        (at ?x)                                 // precondition
        (!walk ?x ?y)                           // primitive task
    )
)

(:operator (@set-cash ?old ?new)                // operator name
    ((have-cash ?old))                          // to remove
    ((have-cash ?new))                          // to add
)
\end{verbatim}
\caption{HTN planning domain}
\label{fig:htnexample}
\end{figure}

Matching preconditions is similar to the matching process of Prolog
\cite{bratko}. Note that the planner will always try to decompose the highest
branch first.  If the preconditions are met, it will decompose this branch
further until it either cannot decompose any further, or a precondition fails.
In the former case the planner would try to find a reduction for the compound
task \verb|drive-taxi ?x ?y|, which is not defined in this example. In the
latter, the planner will backtrack and try the second branch in this example
\verb|branch_walking|.
\begin{figure}[!ht]
\centering
\begin{verbatim}
1.  ((at downtown)
2.   (have-cash 30)
3.   (distance downtown uptown 40))
\end{verbatim}
\caption{HTN initial state}
\label{fig:htnstate}
\end{figure}

\subsection{Formal Description}
This section defines the syntax and semantics used in the HTN planner. It uses
the same first-order-logic definitions of variable and constant symbols,
function and predicate symbols, terms, atoms, conjuncts, most-general unifiers
(mgu's) as SHOP \cite{shop}, an HTN-planner developed in Java. However, for the
purpose of this thesis, some additional definitions are introduced.

\subsubsection{State}
A \emph{state} is a set of ground atoms, and an \emph{axiom set} is a set of
Horn clauses\footnote{Note that the HTN-planner of Killzone 3 does not support
axioms.}. If $S$ is a state and $X$ is an axiom set, then $S \cup X$
\emph{satisfies} a conjunct $C$ if there is a substitution $u$ (called a
\emph{satisfier}) such that $S \cup X \vDash C^u$. $u$ is a \emph{most general
satisfier} if a satisfier $v$ can be expressed by $v = uw$, where $w$ is
another substitution. Note that the state $s$ in RL differs from state $S$ in
the HTN-planner. $s$ defines world properties which are static during an
episode, whereas $S$ is an internal state within the HTN-planner that is
dynamic and thus can change during an episode.
\subsubsection{Task}
A \emph{task} is a list of the form $(s~t_1~t_2~\dots~t_n)$, where $s$, the
task's name, is a task symbol and $t_1, t_2, \dots, t_n$,  the task's
arguments, are terms. The task is \emph{primitive} if $s$ is a primitive task
symbol and \emph{compound} otherwise.
\subsubsection{Operator}
An \emph{operator}\footnote{In Killzone 3, we \emph{emulated} the operators
that manipulate the world state during planning using specific \emph{call}
functions. Note that these functions are not backtrack-safe and for consistent
world state manipulation, operators should be implemented.} is an expression
that has the form $(\verb|:operator|~h~D~A)$, where $h$ (the \emph{head}) is a
primitive task, and $D$ and $A$ (the \emph{deletions} and \emph{additions}) are
sets of atoms containing no variable symbols other than those in $h$. \\ The
intent of an operator $o = (\verb|:operator|~h~D~A)$ is to specify that $h$ can
be accomplished by modifying the current state of the world to remove every
atom in $D$ and add every atom in $A$. More specifically, if $t$ is a primitive
task and there is an mgu $u$ for $t$ and $h$ such that $h^u$ is ground, then
$o$ is applicable to $t$, and the list $(h^u)$ is a simple plan for $t$. If we
execute this plan in some state $S$, it produces the state $h^u(S) = o^u(S) =
(S - D)^u \cup A^u$.
\subsubsection{Method}
A \emph{method} is an expression that has the form $(\verb|:method|~h~B)$,
where $h$, the method's head, is a compound task, $B = \{b \in B:b = (C~T)\}$
is an \emph{ordered} list of \emph{branches}, where $C$ (\emph{precondition} of
a branch) is a conjunct, and $T$ (the \emph{tail} of a branch) is a task list.
Note that the order of the branches is fixed. \\
The intent of a method $m = (\verb|:method|~h~b_1~\dots~b_n)$ is to specify
that if the current state of the world satisfies one of $C \in b_i$, then $h$
can be accomplished by performing the tasks in $T \in b_i$ in the order given.
More specifically, let $S$ be a state, $X$ be an axiom set, and $t$ be a task
atom.  Suppose there is an mgu $u$ that unifies $t$ with $h$, and suppose $S
\cup X \vDash C^u$. Then $m$ is applicable to $t$ in $S \cup X$, and the result
of applying $m$ to $t$ is the set of tasks lists $R = \{(T^u)^v : \textrm{$v$
is an mgs for $C^u$ from $S$}\}$. Each task list $r \in R$ is a \emph{simple
reduction} of $t$ by $m$ in $S \cup X$.
\subsubsection{Plan}
A \emph{plan} is a list of heads of ground operator instances. If $p$ is a plan
and $S$ is a state, then $p(S)$ is the state produced by starting with $S$ and
executing the operator instances in the order that their heads appear in $p$.
\subsubsection{Planning Problem}
A \emph{planning problem} is a tuple $P = (S,T,D)$, where $S$ is a state, $T$
is a task list and $D$ is a set of axioms, operators and methods. Let
$\Pi(S,T,D)$ be the set of all plans for $T$ from $S$ in $D$. We can define
$\Pi(S,T,D)$ recursively as follows.\\ 
If $T$ is empty, then $\Pi(S,T,D)$ contains exactly one plan, namely the empty
plan. Otherwise, let $t \in T$ be the first task atom, and $R = T - t$, the
remaining task atoms. There are three cases.  $(1)$ If $t$ is primitive and
there is a simple plan $p$ for $t$, then $\Pi(S,T,D) =
\{\textrm{\emph{append}}(p,q) : q \in \Pi(p(S),R,D)\}$.  $(2)$ If $t$ is
primitive and there is no simple plan for $t$, then $\Pi(S,T,D) = \emptyset$.
$(3)$ If $t$ is compound, then $\Pi(S,T,D) =
\bigcup\{\Pi(S,\textrm{\emph{append}}(r,R),D) : \textrm{$r$ is a simple
reduction of $t$}\}$.
 % Background

\chapter{Related Work} % Write in your own chapter title
\label{Chapter3}
\lhead{Chapter 3. \emph{Related work}} % Write in your own chapter title to set the page header
This chapter describes some related AI techniques used in both the gaming
industry and the academic world. These techniques are not all directly related
to the method proposed in this thesis, but also show the current state of the
art within the gaming industry and the academic world.
\section{Game Industry}
\subsection{F.E.A.R.}
First Encounter Assault Recon (F.E.A.R.), a first-person shooter developed by
Monolith Productions and published by Vivendi, was released in 2005. The game's
story revolves around a supernatural phenomenon, which F.E.A.R. -- a fictional
special forces team -- is called to contain. The player assumes the role of
F.E.A.R.'s Point Man, who possesses superhuman reflexes, and must uncover the
secrets of a paranormal menace in the form of a little girl. F.E.A.R. was one
of the first games in which the AI developers were able to separate their
domain language from the problem solver using a technique called Goal-Oriented
Action Planning \cite{fear} with the STRIPS \cite{russell} planner. However,
this system was only applied on the individual level of the NPCs, the level
which controls the reactive behavior of a single NPC.
\subsubsection{Goal-Oriented Action Planning}
F.E.A.R. applies a Goal-Oriented Action Planner (GOAP) for its decision making
logic. An agent in F.E.A.R. constantly selects its most relevant goal to
control its behavior. At each logic time step, the most relevant or best goal (a
desired state of the world) is selected and a sequence of actions (plan) is
constructed which is able to satisfy that goal most effectively. The formalized
process of searching for a sequence of actions that satisfies a goal used in
F.E.A.R. closely resembles an automated planner called STRIPS.
\subsubsection{STRIPS}
STRIPS was developed at Stanford University in 1970. STRIPS is an acronym for
STanford Research Institute Problem Solver \cite{russell} and can be seen as
the predecessor of the HTN-planner. The crucial difference between STRIPS and
an HTN-planner is that in the former, the reasoning process takes place at the
level of the actions (operator space) whereas in the latter the reasoning
process takes place at the level of the tasks (plan space) \cite{htnstrips,
htnstrips2, htnstrips3}. STRIPS consists of goals and actions where goals
describe some desired state of the world we want to reach, and actions are
defined in terms of preconditions and effects. An action may only execute if
all of its preconditions are met.  Each action changes the state (conjunction
of literals) of the world in some way.  The STRIPS planner applied to F.E.A.R.
assigns costs to actions and tries to find a shortest path within the action
space using A* to construct its plan. In order to achieve the desired behavior,
one has to tweak these cost values such that A* finds the ``right'' path. This
turns out to be a precise and tedious task in practice.

\subsection{Halo 2}
Halo 2 is a first-person shooter video game developed by Bungie Studios.
Released for the Xbox video game console on November 9, 2004. The player
alternatively assumes the roles of the human Master Chief and the alien Arbiter
in a 26th century conflict between the human UNSC and genocidal Covenant.
Players fight enemies on foot, or with a collection of alien and human
vehicles.
\subsubsection{Hierarchical Finite State Machine}
The control structure of Halo 2 uses a hierarchical finite state machine (HFSM)
or, more specifically, behavior tree or behavior DAG (Directed Acyclic Graph)
\cite{halo3}. A finite state machine is a behavior model composed of a finite
number of states, transitions between those states and actions similar to a
flow graph. An HFSM imposes a hierarchy on the model, where non-leaf states
make decisions about which children to run and leaf states perform a certain
action, see Figure
\ref{fig:hfsm}.
\begin{figure}[!ht]
\centering
\includegraphics[height=200px]{hfsm.eps}
\caption{Hierarchical Finite State Machine}
\label{fig:hfsm}
\end{figure}
There are two general approaches in the decision making of the non-leaf
behavior and at different times Halo 2 uses them both:
\begin{itemize}
\item{The parent makes the decision based upon the conditions using custom
code.}
\item{The children compete with each other, with the parent making the final
choice based upon child behavior desire-to-run or relevancy.}
\end{itemize}
The different characters in Halo 2 can all have different behaviors. However,
as most basic behaviors are shared, the game uses custom behaviors. Each
character uses the same HFSM scheme, but specific characters trigger different
children. One of the main differences between an HFSM and an HTN is the domain
language.  The domain language of an HFSM is not strictly defined, leaving a
lot of design decisions to the programmer. Whereas the HTN domain language is
well defined, giving a solid framework to apply reinforcement learning on.

\section{Academic Work}
\subsection{High Level Reinforcement Learning in Strategy Games} 
As human players rapidly discover and exploit the weaknesses of hard coded
strategies in games, this paper presents a reinforcement learning approach for
learning a policy that switches between high-level strategies \cite{rl-rts-1}.
The testbed for this paper is \verb|Civilization IV|, a complex strategy
game\footnote{Strategy video games are a genre of video game that emphasize
skillful thinking and planning to achieve victory. They emphasize strategic,
tactical, and sometimes logistical challenges.} in which players evolve a
culture through the ages.
\subsubsection{Model-based Q-learning: Dyna-Q}
Q-learning is a model-free method, meaning it learns a policy directly, without
first obtaining the model parameters (transition and reward functions). An
alternative is to use a model-based method that learns the model parameters and
uses the model definition to learn a policy.
Dyna-Q is a method that can learn the model and the Q-values at the same time.
Thus, the agent learns both the Q-values and the model through acting in the
environment. This model is then used to simulate the environment and the
Q-values are updated accordingly \cite{rl-rts-1}. As the model becomes a better
representation of the problem, the Q-values become more accurate and
convergance will occur more quickly. See algorithm \ref{alg:dynaq}.
\begin{algorithm}
	\caption{\emph{Dyna-Q}$(Q,r,s,a)$: Returns updated Q-values, $Q$}
	\begin{algorithmic}[1]
	\REQUIRE The Q-values, $Q$, immediate reward $r$, state $s$ and action $a$
	\medskip
	\STATE $Q(s,a) \leftarrow Q(s,a) + \alpha(r + \gamma Q(s',a') - Q(s,a))$
	\STATE $P(s'|s,a) \leftarrow \textrm{\emph{updatePAverage}}(s,a,s')$
	\STATE $R(s,a) \leftarrow \textrm{\emph{updateRAverage}}(s,a)$
	\FOR{$i = 0$ \TO $N$}
		\STATE $s' \leftarrow \textrm{\emph{randomPreviouslySeenS}}()$
		\STATE $a' \leftarrow \textrm{\emph{randomPreviouslyTakenA}}(s')$
		\STATE $s'' \leftarrow \textrm{\emph{sampleFromModel}}(s',a')$
		\STATE $r' \leftarrow \textrm{\emph{fromModel}}(s',a')$
		\STATE $Q(s',a') \leftarrow Q(s',a') + \alpha(r + \gamma Q(s'',a'') - Q(s',a'))$
	\ENDFOR
	\RETURN $Q$
	\end{algorithmic}
\label{alg:dynaq}
\end{algorithm}
First the regular Q-learning update is performed and the probability and reward
models are updated as averages given the new information. The model sampling
occurs in the for-loop. For some designated number of iterations $N$ the model
is sampled and the Q-values are updated accordingly.

\subsubsection{Civilization as MDP}
The statespace is defined as a set of four state features\footnote{State
features can improve the speed of learning assuming the individual features are
independent of eachother.}: population difference, land difference, military
power difference and remaining land. These features, $f_1,\ldots,f_4$ are
discretized into three different values:
\begin{displaymath}
f_i = \left\{ \begin{array}{ll}
2 & \textrm{if \emph{diff}} > 10\\
1 & \textrm{if} -10 < \textrm{\emph{diff}} < 10\\
0 & \textrm{if \emph{diff}} < -10
\end{array} \right.
\end{displaymath}
Where \emph{diff} represents the difference in value between the agent and the
opponent.\\
An action is a choice of strategy that is built-in into the game. The action
space was limited to four different actions.  Each action represents a
different type of play of the game: Agressive and expansive, financial and
organized, etc.\\
The immediate reward is defined as the step based score of the game. That is
the difference of the agent's total score and the opponent's total score.
\\\\
Although this approach works reasonably well for the Civilization game, in a
first-person shooter (FPS) game, the gameplay is much more fast-paced and
changes occur in rapid succession. In order to cope with this properly, more
detailed world information needs to be encoded within the state. The main
challenges can therefore be found in properly modelling the FPS as an MDP
without blowing up the state-action space. Another problem is the partial
observability within the FPS, the enemy location can only be acquired through
scouting the environment by individual agents.


\subsection{HTN for Encoding Strategic Game AI}
The paper presents a case study for HTN-Planning on a strategic level in the
game called Unreal Tournament\footnote{A first-person shooter video game
co-developed by Epic Games and Digital Extremes. It was published in 1999 by GT
Interactive.} (UT) \cite{ut}. The game mode to which the case study is applied
is called domination. In domination, there are fixed locations in the world
that can be captured by letting a team member step into a location. The team
gets a point for every five seconds that each domination location remains under
the control of that team. The game is won by the first team that gets a
pre-specified amount of points.

\subsubsection{Strategies}
Two different strategies were encoded in the HTNs. The first strategy is called
\emph{Control Half Plus One Points}. This strategy selects half plus one of the
domination locations and sets bots to capture these points. The second strategy
\emph{Control All Points} requires that the team consists of at least two
members. It calls for two members to capture all domination locations and
patrol between them. The remaining team members are assigned to search and
destroy tasks.

\subsubsection{Architecture}
In order to compute the HTN for the grand strategies the Java based SHOP
Planner\cite{shop} is applied. An event-driven program encoded in the Javabot
FSMs allows the individual bots to react to the environment while contributing
to the grand task at hand. 

Differentiating between strategies appears to be performed manually instead of
using some adaptive method or heuristic.

\subsection{Neural Networks and Evolving AI in FPS Games}
Other approaches to optimize bot behavior in FPS games involve the application
of neural networks and evolutionary algorithms. Although there have been
successfull attemps in applying these methods \cite{ENNQ3,hclfps,ECBU}, the
main problem remains the loss of control over the bots as they converge to the
(local) optimal, without some form of granuality. Aside from that, neural
networks require many offline training rounds and it can be difficult to
understand what is going on when they become large. Making them hard to debug
during development.
 % Related work

\chapter{Domain} % Write in your own chapter title
\label{Chapter4}
\lhead{Chapter 4. \emph{Domain}} % Write in your own chapter title to set the page header
\section{Guerrilla}
Guerrilla is a game development studio, based in the heart of Amsterdam, the
Netherlands. It was formed at the beginning of 2000 as a result of a merger
between 3 separate Dutch-based developers. The company now employs 130
developers, designers and artists, encompassing 20 different nationalities. The
first game released by Guerrilla, Shellshock: Nam '67 was developed for the PC,
Xbox and PlayStation 2 and published by Eidos Interactive. In 2004, Guerrilla
signed an exclusive deal with Sony Computer Entertainment. Under that deal,
Guerrilla developed games exclusively for Sony's consoles (PlayStation 2,
PlayStation 3 and the PlayStation portable). After the release of Killzone for
the PlayStation 2 (2004), the company was acquired by Sony Computer
Entertainment in 2005. It went on to release Killzone: Liberation for
PlayStation Portable (2006), and KillZone 2 (2009). At the time of writing this
thesis, the company just released a new title for the PlayStation 3 called
Killzone 3.

\section{Killzone 3}
Killzone 3 (KZ3) is a first-person shooter (FPS) game. Figure
\ref{fig:killzone} shows a typical ingame scenario from KZ3 as viewed by the
player.
\begin{figure}[!ht]
\centering
\includegraphics[height=230px]{kz3fps.eps}
\caption{Killzone 3 ingame first-person view}
\label{fig:killzone}
\end{figure}
Like most 3D shooters, KZ3 offers a variety of playing possibilities with
friends or online. It distinguishes three different modes: Singleplayer,
cooperative play and multiplayer. This thesis focuses on the multiplayer mode.

\subsection{Multiplayer}
A Multiplayer online game is a multiplayer video game which is played via a
game server over the Internet, with other human players around the world.
Players either compete against each other (individually or in teams/clans) or
cooperate with each other against a common enemy (e.g. an AI). In contrast to
singleplayer mode, the game is played on one single stage only. Since these
games are not centered around one player, when a player dies, the game is not
restarted. Instead, these games continue and the player that died will have the
opportunity to rejoin the game. \\
A multiplayer game mode is defined by a set of rules and regulations that
specify game objectives, win/lose scenarios and conditions for scoring and
ranking on team and individual basis. For any game mode, points are rewarded
for killing enemies. However, many game modes define multiple different
(primary) objectives and therefore require different strategies to win. The
nature and amount of objectives vary among the different game modes and can
even be different for the opposing teams. For instant in the symmetric game
mode ``Bodycount'', the teams have to kill as many players in the opposing team
as possible, where each kill gains a point. In the game mode ``Assasination'',
a non-symmetric mode, team one has to assassinate a single key player in team
two, which team two has to defend. For this thesis, we will create strategies
for the symmetric game mode ``Capture and Hold''. 

\subsubsection{Capture and Hold}
In this game mode, there are three key areas on the map that can be captured
by either team. To gain control over an area, players must stand within the
capture radius of the area and must make sure that no enemy soldiers are within
capture range of that same area during the capture process. If a team succeeds
in keeping away the enemy long enough, the area will change ownership to the
capturing team. Once an area is captured the team receives points for each
time step the area is under their control.  If the maximum amount of points is
gained (defined by the game) the team wins or, alternatively, if the time limit
is reached the team with the most points wins. When both teams have the same
amount of points the game results in a draw.

%\subsubsection{Bodycount}
%\subsubsection{Assasination}
%\subsubsection{Search and Destroy}
%\subsubsection{Search and Retrieve}
%\subsubsection{Capture and Hold}

\subsection{Multiplayer AI Design Overview}
Singleplayer and coop AI differ greatly from multiplayer AI. The reason for
this difference is that NPCs found in singleplayer games, have a different role
from those in multiplayer games.\\
In order to put the multiplayer bot behavior -- at a strategic level -- in
perspective, this section gives a general overview of the entire AI
architecture \cite{killzone2,killzone2-presentation}.  Figure
\ref{fig:designoverview} shows a simplified overview of the AI hierarchy as
defined in Killzone 3.
\begin{figure}[!ht]
\centering
\includegraphics[height=250px]{DesignOverview.eps}
\caption{Simplified design overview of the AI architecture}
\label{fig:designoverview}
\end{figure}
The top layer shows the decision system at the strategic level, often referred
to as commander or general\footnote{``\emph{Strategy without tactics is the
slowest route to victory. Tactics without strategy is the noise before
defeat.}'' -- Sun Tzu.}. This is the level where the method, proposed in
Chapter \ref{Chapter5}, is focused on.  The subsections below will explain
each layer in more detail (though still at high level). Note that both the
commander and the group are concepts within the AI architecture, they do not
represent actual entities in the virtual world.
\subsubsection{Individual}
The individuals define the actual NPCs. They observe their surroundings through
visual and auditive stimuli modeled after how humans observe the world. This
prevents stupid mistakes such as an NPC ``mysteriously'' knowing that an enemy
is behind him. These observations about the world are stored in a world fact
database. The world fact database is local and different for each individual
and used by the HTN-planner to create behavorial plans using orders.  These
orders include reloading weapon, firing weapon, going in cover, blind fire,
etc.
\subsubsection{Group}
A group, as the name implies, defines a set of individuals. These individuals
form a small military unit often referred to as a squad\footnote{The terms
squad and group are used interchangeable throughout this thesis.}.  The group
is responsible for e.g. defending an area or capturing a strategic point.
Groups are both created and controlled at the commander layer using commands.
Each group also uses a unique world fact database which stores information such
as where each individual is located and which group it belongs to. The squad is
responsible for the coherency of the individuals that belong to it during
movement.
\subsubsection{Commander}
The commander is the top layer in the AI. Its actions are performed on the
strategic level. The commander is responsible for the following tasks:
\begin{itemize}
\item{Squad creation}
\item{NPC to squad allocation}
\item{Commands for squads}
\end{itemize}
The commands that can be sent to squads include: Going to a certain waypoint,
attacking an entity, defending a marker or entity. The squads, in turn, can
report back the status of their progress. These commands form the primitive
actions which, when put in the right sequence, form a sensible plan. We
expanded the HTN architecture to support the commands above for the commander
such that we can encode a strategy in a domain. Given this domain, the
HTN-planner can construct a plan at the start of e.g. a \emph{capture and hold}
game. Figure \ref{fig:ch-example} shows how the command sequence of a squad
capturing an area is build.
\begin{figure}[!ht]
\centering
\begin{verbatim}
(:method (capture_areas ?inp_area_list)
  (branch_areas_captured
    (and (call eq (call get_list_length ?inp_area_list) 0)
         (= ?squad_index (call get_last_created_squad_index))
         (= ?squad (call get_squad ?squad_index))
    )
    (    (!end_command_sequence ?squad)
    )
  )
  
  (branch_capture_area
    (and (= ?area_index (call get_list_item ?inp_area_list 0))
         (= ?area_list (call remove_list_item ?inp_area_list 0))
         (= ?squad_index (call get_last_created_squad))
         (= ?squad (call get_squad ?squad_index))
    )
    (    (!order_squad_custom ?squad capture ?area_index)
         (!order_squad_custom ?squad advance ?area_index auto)
         (capture_areas ?area_list)
    )
  )
)
\end{verbatim}
\caption{Area capture example domain}
\label{fig:ch-example}
\end{figure}
The method requires a list of areas indices as argument and orders a squad to
capture the area and advance. While there are still areas that require
capturing, the method assigns a squad. Finally an ``end of command sequence''
is given.
\\\\
Applying the HTN-planner at this layer separates the domain language from the
problem solver. This separation allows for the application of reinforcement
learning integrated within the HTN-planner. Integrating the RL algorithm on
this level enables machine learning the best strategies across multiple game
modes against various fixed opponents.
\subsubsection{C++ Strategy Implementation}
Currently the strategical logic is implemented in C++. This is done using an
objective based approach. An objective is implemented as a C++ class which
defines the desired amount of bots and squads to be accomplished. Some examples
of objectives are:
\begin{itemize}
\item{Attack Entity}
\item{Defend Marker}
\item{Escort Entity}
\end{itemize}
Every logical update, the objectives and their importances are computed using
hardcoded heuristics and squads get assigned or reassigned to these objectives.
This process continues until the game is over.
 % Domain

\chapter{Approach}
\label{Chapter5}
\lhead{Chapter 5. \emph{Approach}}
This Chapter introduces the learning algorithm and the hand crafted strategies
for Killzone 3's multiplayer game mode \emph{capture and hold}. The proposed
method is a modified version of the HTN planner combined with a simple form of
reinforcement learning, a basic description follows. At the start of a
multiplayer game of Killzone 3 (e.g. \emph{capture and hold}), a strategy is
constructed using the HTN planner which was either selected at random during
exploration or greedily during exploitation. The strategy is executed and at
the end of the round or episode, its reward is observed and the value of the
weight belonging to the strategy is updated accordingly. So each episode a
single action is chosen and executed, depending on environmental variables,
making this a contextual bandit problem.

\section{Learning AI}
As stated in Chapter \ref{Chapter2}, the ordering of branches of the methods in
the original HTN planner is fixed. This fixed ordering makes sense if there is
a clear order in which a planning problem should be decomposed given a state
$S$ and an axiom set $X$. At the strategic level of the Killzone 3 AI, there
exist methods for which there is no clear ordering of their branches
beforehand. That is, given some method $m = (\verb|:method|~h~b_1~\dots~b_n)$
from the strategic domain, the order in which $C$ from $(C~T) \in b_i \forall
i$ is entailed for some mgu $u$ that unifies $t^u \in T$ with $h$ such that $S
\cup X \vDash C^u$, cannot be defined in terms of $S$ and $X$ alone. This
section proposes an algorithm that can adapt the ordering by applying
reinforcement learning.

\subsection{Branch Ordering and Selection}
%TODO: announce that only one learnable method can be used in a branch
By assigning weights (or values) $w$ to each branch $b_{im}$ in a method $m$ we
can sort the branches on their respective weight in descending order and execute
the first branch whose preconditions are met. A branch is a tuple $b_{im} =
(C_{im}~T_{im}~w(s)_{im})$, where $C_{im}$ holds the preconditions, $T_{im}$
holds the task list and $w(s)_{im}$ is a weight for a certain contextual state
$s$. As explained in chapter \ref{Chapter2}, state $s$ defines world properties
which are static during an episode (the map onto which the episode is played
and the team or faction), whereas $S$ is a world state defining potential
non-static properties during an episode. During exploration by e.g.
$\epsilon-$\emph{Greedy}, branches are randomly selected, while during
exploitation branches are sorted by their weight in descending order and the
first branch for which precondition $C_{im}$ is met can be further decomposed.
\\ \\
Given the Q-learning update rule in \ref{eq:qlearn}, we will now present how we
apply an adapted version in our algorithm. As our problem is modelled as a
contextual bandit problem, we only chose one action per episode:
\begin{eqnarray}
Q_t(s,a) & = & (1-\alpha_t)Q_{t-1}(s,a) + \alpha_t\Big[r + \gamma max_{a'}~Q_{t-1}(s',a')\Big] \\
         & = & (1-\alpha_t)Q_{t-1}(s,a) + \alpha_t\Big[r + 0\Big] \\
         & = & Q_{t-1}(s,a) - \alpha_t Q_{t-1}(s,a) + \alpha_t r \\
         & = & Q_{t-1}(s,a) + \alpha_t\Big[r - Q_{t-1}(s,a)\Big]
\end{eqnarray}
As such there is no transition from $s$ to $s'$ within an episode, reducing the
right term in the square brackets to $0$. At the end of an episode, a branch
update occurs by observing the immediate reward $r$ on the current leafnode
after which the highest value is propagated upwards to the rootnode:
\begin{equation}
w(s)_{im} = \left\{ \begin{array}{ll}
w(s)_{im} + \alpha\Big[r - w(s)_{im}\Big] & \textrm{if $w(s)_{im}$ is a leafnode}\\
\textrm{\emph{max}}_{c \in \textrm{\emph{children}}(m)} w(s)_{ic} & \textrm{otherwise}
\end{array} \right.
\label{eq:update}
\end{equation}
Where $\alpha$ denotes the learning rate, $c$ denotes a child method under
parent $m$ and $i$ the branch index, with $im$ or $ic$ uniquely identifying the
action. Thus equation \ref{eq:update} successfully implements the reinforcement
learning update rule within an hierarchical environment.

\subsection{Pruning the Action Space}
As stated in chapter \ref{Chapter2}, the HTN is able to define the action set
for RL. Instead of allowing all possible sequence of actions to be chosen, as
is the case in regular reinforcement learning, a plan (specific sequence of
primitive tasks) is encoded using the HTN. This greatly reduces the action
space, allowing only sensible sequences of tasks to be executed as defined by
the prior knowledge of the developer or expert.

The pruning of the action space does come at a potential cost. Figure
\ref{fig:htngraph} shows a graphical depiction of the HTN planner in action,
the nodes represent (composite) methods and the edges represent the branches of
a method. The planner starts at the root node and tries to decompose the
\emph{ordered} branches from the highest weight to the lowest at each level in
the tree in a depth first manner.
\begin{figure}[!ht]
\centering
\includegraphics[height=150px]{htngraph.eps}
\caption{HTN Graph}
\label{fig:htngraph}
\end{figure}
In this case, the most left leaf-node cannot be reduced as its preconditions
are not met. The planner performs a backtrack and tries to decompose its
sibling, which is of weight $2$, whereas the best leaf-node would be the most
right node with a weight of $3$. This is the potential pitfall that is the
result of pruning the action space by HTN encoding and it is up to the prior
knowledge of the developer to ensure a proper encoding. The sibling nodes
should have a contextual relation with each other that enforces equal
constraints. In this case, the parent node of the node with weight $5$ should
have this constraint resulting in the right most node with weight $3$.

\section{Implementation}
The method is implemented using two algorithms \emph{execute-plan} and
\emph{seek-plan}. The method first tries to find a plan, executes it
and performs the weight update.
\subsection{Execute Plan}
Algorithm \ref{alg:executeplan} first calls subroutine \emph{seek-plan} which
returns a plan $p$ and its traversed path of weights $W$ (in reverse order) in
the form of a list that require updates. If the method corresponding to
$w(s)_m \in W$ is a leaf-node, the immediate reward update is applied,
otherwise the \emph{max} weight of the children of $m$ is assigned to $w(s)_m$.
This propagates the highest weight to the root node, making sure the best path
is selected during exploitation. The method $c$ variable is used for keeping
track of the child method.

\begin{algorithm}
	\caption{\emph{execute-plan}$(S,T,D)$: Executes a plan and updates the weights}
	\begin{algorithmic}[1]
	\REQUIRE The state $S$, task list $T$ and $D$ a set of operators, axioms and methods
	\medskip
	\STATE $(p,W) \leftarrow$ \emph{seek-plan}$(S,T,D,nil,nil)$
	\STATE observe world state $s$
	\STATE run episode with plan $p$, observe reward $r$
	\STATE method $c \leftarrow nil$
	\FORALL{$w(s)_m \in W$}
		\IF{\emph{is-leaf}$(m)$}
			\STATE $w(s)_m \leftarrow w(s)_m + \alpha [r - w(s)_m]$
			\STATE $c \leftarrow m$
		\ELSE
			\STATE $w(s)_m \leftarrow \textrm{\emph{max}}(\{w(s)_c : w(s)_{ic} \in \textrm{\emph{branches}}(c)\})$
			\STATE $c \leftarrow m$
		\ENDIF
	\ENDFOR
	\end{algorithmic}
\label{alg:executeplan}
\end{algorithm}
Next, the episode (simulation) is executed with plan $p$ applied and reward $r$
is observed at the end of the episode. This reward is a numerical value and
defined as follows:
\begin{equation}
r = m - e,
\label{eq:reward}
\end{equation}
where $m$ defines the mission points of the current team and $e$ the mission
points of the enemy.\footnote{The actual formula for computing mission points in
Killzone 3 varies per game mode. In the case of \emph{capture and hold}, mission
points are assigned per timestep to teams that control captureable areas.}
This results in a reward $r \in [-50,50]$, where $r < 0$ indicates a loss and
$r > 0$ indicates a victory and $r = 0$ indicates a draw.  Finally the
traversed weight path is updated according to equation \ref{eq:update}.

\subsection{Seek Plan}
The subroutine \emph{seek-plan} shown in Algorithm \ref{alg:seekplan} returns a
tuple $(p,W)$, where $p$ is the plan found and $W$ the corresponding list of
weights of the traversed path in reverse order.
\begin{algorithm}
	\caption{\emph{seek-plan}$(S,T,D,p,W)$: Returns a plan and the weigths of the learnable methods along the path}
	\begin{algorithmic}[1]
	\REQUIRE The state $S$, task list $T$ and $D$ a set of operators, axioms and methods, $p$ the plan, $W$ the set of weights of the learnable methods found in the traversed path
	\medskip
	\IF{$t = nil$}
		\RETURN $(p,W)$
	\ENDIF
	\STATE $t \leftarrow$ first task in $T$, $R \leftarrow T - t$
	\IF{$t$ is primitive}
		\IF{there is a simple plan $q$ for $t$}
			\RETURN \emph{seek-plan}$(q(S),R,D,\textrm{\emph{append}}(p,q),W)$
		\ELSE
			\RETURN $\bot$
		\ENDIF
	\ELSE
		\STATE observe callstack $c$
		\FORALL{$m \in D$ that can reduce $t$ in $S$}
		\IF{$m$ is learnable}
			\STATE sort branches on $w_{im_c}$ in descending order
		\ENDIF
		\STATE $r \leftarrow$ reduction of $t$ using $m$ in $S$
		\STATE $(p', W') \leftarrow \textrm{\emph{seek-plan}}(S, \textrm{\emph{append}}(r,R),D,p,\textrm{\emph{append}}(w_{im_c},W))$
		\IF{$p' \neq \bot$}
			\RETURN $(p', W')$
		\ENDIF
		\ENDFOR
	\ENDIF
	\end{algorithmic}
\label{alg:seekplan}
\end{algorithm}
The algorithm is an extended version of the HTN planner which was already
available in Killzone 3. It recursively decomposes the given task list $T$
using state $S$ and the set of operators $D$ until the entire plan consists of
primitive actions only. Along the way it stores the weights $w \in W$ that
require updating. Since a method $m$ can be used in different domains, the
corresponding runtime callstack $c$ of method $m$ is used to discriminate
between the different contexts. Thus the callstack and method provide a unique
key for a weight.

\section{Strategies}
For this thesis, three main strategies were developed for the game mode
\emph{capture and hold} using three areas, namely: Steam Roll, Capture and
Split, Divide and Conquer. Each of the strategies contains multiple variations
which determine area capturing sequences, number of squads and relative sizes
of the squads. The domain file for the \emph{capture and hold} strategies is
listed in appendix \ref{AppendixA}.
\subsection{Steam Roll}
The strategy \emph{steam roll} is depicted in Figure \ref{fig:steamroll}. In
this strategy, a single squad \verb|A| is created and sequentially traverses
each of the areas that require capturing. In this case, the squad captures the
areas in the following sequence: $1,2,3$. Variations on this strategy are the
different capture sequences of the areas, indicated by \emph{areaXYZ}, where
\emph{XYZ} can be any permutation of $1,2,3$.
\begin{figure}[!ht]
\centering
\includegraphics[height=200px]{SteamRoll.eps}
\caption{Strategy: Steamroll}
\label{fig:steamroll}
\end{figure}
The created squad is indeed very strong and will most likely capture the area
it's going for. However, its squad can only capture a single area at a time,
making it a slow strategy. Secondly, the captured areas will be left undefended
completely.

\subsection{Capture and Split}
This strategy creates three squads instead of a single squad as is the case in
\emph{steam roll}. First the three squads capture a single area together. Next
a single squad stays behind and the remaining squads go to the second area and
capture it, where the second squad also stays behind and the third squad
finally captures the last area. 
\begin{figure}[!ht]
\centering
\includegraphics[height=250px]{ConquerAndDivide.eps}
\caption{Strategy: Capture and Split}
\label{fig:candd}
\end{figure}
This strategy poses two variables along which it can differ, capture sequences
as in \emph{steam roll}, and the various number of individuals per squad. A
specific strategy from \emph{capture and split} is thus defined as
\emph{areaXYZ\_sqdABC}, where \emph{areaXYZ} is equally defined as in
\emph{steam roll} and \emph{sqdABC} defines the distribution of the individuals
among the squads. For instance \emph{sqd121} means that the individuals are
divided over the squads as $\{\frac{1}{4}, \frac{2}{4}, \frac{1}{4}\}$. See
Figure \ref{fig:candd} for an example of this strategy.

\subsection{Divide And Conquer}
The last strategy is shown in Figure \ref{fig:dandc}. \emph{Divide and conquer}
divides its forces into three squads and tries to capture each of the areas in
parallel. This strategy also has the ability to vary along the squad
distributions like \emph{capture and split}. A variation of this strategy is
thus defined as \emph{sqdABC}.
\begin{figure}[!ht]
\centering
\includegraphics[height=220px]{DivideAndConquer.eps}
\caption{Strategy: Divide and Conquer}
\label{fig:dandc}
\end{figure}
This type of strategy, when successful, will capture each of the areas the fastest.
However, each with a much weaker force that can be overrun by the enemy with less
effort than the previous strategies.
 % Method

% Chapter 6 
\chapter{Experiments and Results} % Write in your own chapter title 
\label{Chapter6} \lhead{Chapter 6. \emph{Experiments and Results}} % Write in your own chapter title to set the page header
In this chapter we empirically analyze the method using the \emph{Killzone 3}
environment. We devised three different experiments that will answer our
questions posed in the first chapter:
\begin{enumerate}
\item{Can we encode good strategies into a hierarchical task network?}
\item{Is it possible to learn the best strategy with respect to a fixed opponent?}
\item{Is it possible to adapt to a different fixed opponent?} 
\end{enumerate}
The first is a training run against the baseline, which is the current C++
strategy version, on various maps. The second is a comparison between the
method and averaged reward data from the strategies against the baseline.
Finally we apply the trained version from the first experiment to a different
fixed opponent and observe its adaptive capabilities.


\section{Settings}
For our experiments we used three different multiplayer maps MP01, MP02 and
MP03. Most experiments were run on MP01, also known as \emph{Corinth
Highway}\footnote{Killzone 3's exo mounting was disabled for both teams as exos
were introduced after the strategies were developed.}. Figure \ref{fig:mp01}
shows a top view of MP01, on the left we can see the ISA base and on the right
the HGH base. In the center we can see the capturable areas CH1, CH2 and CH3.
\begin{figure}[!ht]
\centering
\includegraphics[width=420px]{MP1_Marked_WZ.eps}
\caption{Corinth Highway (MP01)}
\label{fig:mp01}
\end{figure}
\begin{figure}[!ht]
\centering
\includegraphics[width=420px]{MP2_Marked_WZ.eps}
\caption{Pyrrhus Crater (MP02)}
\label{fig:mp02}
\end{figure}
\begin{figure}[!ht]
\centering
\includegraphics[width=420px]{MP3_Marked_WZ.eps}
\caption{Bilgarsk Boulevard (MP03)}
\label{fig:mp03}
\end{figure}
Figure \ref{fig:mp02} and \ref{fig:mp03} show two more maps, with the
captureable areas more widespread across the map. The detached images in figure
\ref{fig:mp03} represent different floors of the map, in this case both the ISA
and HGH bases are located on the first floor.  As can be seen in all the
figures, the maps are symmetric with respect to the starting positions for both
teams in order to ensure a fair battle.  Each episode had a timelimit of $10$
minutes and $16$ bots divided over two factions or teams ($8$ versus $8$).


\section{Training}
The method was trained against the baseline, which is the current C++ strategy
implementation, for $150$ episodes on three different multiplayer maps averaged
over $3$ runs. During training, the learning-rate $\alpha$ was set to $1/4$ and
exploration-rate $\epsilon$ started at $1.0$ and gradually decayed to $0.1$
after each episode.\footnote{To compute the decay-rate $d$ we used $d =
(\epsilon_f / \epsilon_s)^{1/E}$, where $\epsilon_s$ is the start
exploration-rate, $\epsilon_f$ is the final exploration-rate and $E$ is the
amount of episodes.} Figure \ref{fig:training} shows the maximum reward gained
after each episode for the three different multiplayer maps. As stated in
chapter \ref{Chapter4}, the reward is defined as $r = m - e$. In the case of
\emph{capture and hold} it represents the amount of time capturable areas were
under our control minus the areas that were under enemy control. Thus when $r >
0$ indicates a win and $r = 50$ is the maximum score achievable, meaning the
capturable areas were never under enemy control.
\begin{figure}[!ht]
\centering
\includegraphics[height=250px]{training.eps}
\caption{Method vs Baseline}
\label{fig:training}
\end{figure}
By examining the weights corresponding to this graph, it showed that on all
three maps the \emph{divide and conquer} substrategies are most successful.
This is probably due to the parallel nature of the strategy. As each area is
assigned a designated squad at start and points are gained per timestep for
captured areas.


\section{Comparison}
To determine how well the trained weights reflect the real success of the
strategies, we compare it to the averaged rewards for each of the individual
strategies against the baseline.  Figure \ref{fig:weights} shows the weight
distribution obtained for the different strategies during training against the
baseline on MP01 and figure \ref{fig:avg} shows the rewards per strategy
averaged over $10$ episodes. 
\begin{figure}[ht]
\centering
\subfigure[Weight distribution]{
\includegraphics[height=130px]{weights-baseline.eps}
\label{fig:weights}
}
\subfigure[Averaged rewards]{
\includegraphics[height=130px]{real.eps}
\label{fig:avg}
}
\label{fig:comparison}
\caption[Comparison]{\subref{fig:weights} shows the weight distribution after
training against the baseline. \subref{fig:avg} shows the average rewards of
the strategies.}
\end{figure}
From these results we can conclude that the method is consistent with the
averaged rewards w.r.t. the \emph{divide and conquer} strategies (indicated in
blue). For the other strategies, bigger differences can be observed as not
every strategy is explored as thoroughly due to the decreasing
exploration-rate. Furthermore, these differences are the result of fluctuations
in the rewards of the strategies itself per episode. Some episodes the strategy
performed better than others.  These fluctuations can also be seen in figure
\ref{fig:training}, e.g. the red line from episode $100$ to $150$ jumping up
and down between $40$ and $50$. The \emph{stream roll} type strategies were
most ``unstable'', e.g. \emph{area021} varied from $-24$ to $18$ against the
baseline.\\\\
The figure below shows the weight distribution of two training runs on the map
MP02 (\emph{Pyrrhus Crater}). MP02 was chosen as the capture points are further
away from eachother, see figure \ref{fig:mp02}. 
\begin{figure}[ht]
\centering
\subfigure[ISA]{
\includegraphics[height=130px]{mp02-isa.eps}
\label{fig:isa}
}
\subfigure[HGH]{
\includegraphics[height=130px]{mp02-hgh.eps}
\label{fig:hgh}
}
\label{fig:isahgh}
\caption[HGH vs ISA]{\subref{fig:isa} shows the weight distribution after
training against the baseline on ISA side. \subref{fig:avg} shows the weight distribution after
training against the baseline on HGH side.}
\end{figure}
As both factions always start on the same side of the map, we expect to see
some mirrored behavior in the area capturing sequence for \emph{capture and
split} and \emph{steam roll}. The substrategies of \emph{divide and conquer}
barely show a difference as they capture each area simultaniously. On the left
figure \ref{fig:isa} the ISA faction was trained against the baseline and on
the right figure \ref{fig:hgh} the HGH faction was trained against the
baseline. A clear mirrored behavior can be seen in \emph{area201\_sqd212} and
\emph{area201\_sqd111}. Both start at area $2$ and the ISA faction first
captures area $1$, whereas the HGH faction first captures area $0$. This
mirrored behavior is however not shown for the \emph{steam roll} sub strategies
as they are more unstable as stated above.


\section{Adaptability}
To determine the adaptability against a different fixed opponent we chose the
best strategy according to figure \ref{fig:weights}, strategy \emph{sqd121} as
the new adversary. For this experiment $\alpha$ was set to $1/3$ to allow for
more agressive changes and $\epsilon$ was fixed at $1/3$ in order to prevent
getting stuck in local optima. Figure \ref{fig:sqd121} shows the new weight
distribution after $20$ episodes against \emph{sqd121} on MP01.
\begin{figure}[ht]
\centering
\subfigure[Against baseline]{
\includegraphics[height=130px]{weights-baseline.eps}
\label{fig:baseline}
}
\subfigure[Against sqd121]{
\includegraphics[height=130px]{area210_sqd111.eps}
\label{fig:sqd121}
}
\label{fig:adaptability}
\caption[Adaptability]{\subref{fig:baseline} shows the weight distribution
after training against the baseline. \subref{fig:sqd121} shows the weight
distribution after adapting to fixed opponent sqd121.}
\end{figure}
Most \emph{stream roll} strategies dropped further below zero, indicating more
loss and the \emph{divide and conquer} strategies hover around zero as they
are equally strong.  The strategies \emph{area201\_sqd212} and
\emph{area210\_sqd111} pose decent counters and show that the algorithm adapted
to the different fixed opponent.
 % Experiments & Results

\chapter{Conclusions} % Write in your own chapter title
\label{Chapter7}
\lhead{Chapter 7. \emph{Conclusions}} % Write in your own chapter title to set the page header
In this thesis, we proposed a method that maintains a balance between control
by developers and autonomous, learning AI. The method is a novel combination of
hierarchical task network planning and reinforcement learning. Various
handcrafted strategies were encoded in an HTN within a commercial game
environment called \emph{Killzone 3}.\\
We showed that it is possible to learn the best strategy with respect to a
fixed opponent and that, when a different fixed opponent is introduced, the
method has the ability to change its counterstrategy accordingly. Initially the
method can be trained offline against a baseline to bootstrap it e.g. before
shipment. From that basis, it has the ability to learn against a different
fixed opponent as humans apply their strategies against it.

What was somewhat surprising is the good performance of the simple handcrafted
strategies against the baseline, which is the current C++ version in
\emph{Killzone 3}. We think this is because the C++ baseline has the ability to
change its priorities during an episode and often flip flops between them
without first finishing one, whereas our method selects a single strategy at
the start of an episode using the HTN-planner and sticks to this strategy
during the entire episode. Details like how to approach a certain area with a
squad do vary between episodes, but the high level strategy remains the same.
Without proper heuristics that can aid in making decisions on when to
(partially) switch your strategy or sufficient information about the
environment encoded in the state without blowing up the state space, it might
be better to stick to the plan constructed at start.
A potential pitfall lurks in the design of the hierarchical task network. As
stated in chapter \ref{Chapter5}, when a branch is unable to decompose due to
failing preconditions the sibbling branch that gets decomposed because of the
recursive nature of the algorithm might not be optimal.
Currently the method does not support multiple learnable tasks in the tasklist
within a single branch. This would cause several values to return under a
branch and the method has no way of coping with that. Although the current
strategies do not require this type of structure, more complex strategies
might. A possible solution would be to apply some linear function to the values
returned, but this might not be as trivial as it sounds in terms of desired
behavior.

Some other points for future work are the following. Currently the static state
holds the faction and the map on which the episode is played. It might be
interesting to add some features about the enemy which can discriminate between
the strategies the enemy plays. Features like the amount of enemy squads and
the size of the squads, or how agressive the enemy is during attacks. These
features could paint a profile of the enemy that allows for more specific
behavior which require less adjustments during online learning resulting in
faster convergence.\\
Another area that might be interesting for further investigation is to treat
the problem like a full MDP, thus moving away from the contextual bandit
approach. The main challenge would lie in defining the state space such that it
will not blow up i.e. determining all the relevant variables that describe the
environment.


% pros/discussion
%  - see three questions at intro answered in results (*)
%  - surprising that strategies won against c++ version (*)

% cons/discussion
%  - no replanning/adjustments during episode, pre plan everything on highest level
%  - pitfalls of the hierarchy w.r.t. suboptimality 
%  - exploration in hierarchies, long test times, tweaking alpha/epsilon
%  - one learnable method per level in the hierarchy

% future work
%  - opponent modeling, store preferences in state
%  - replanning during episodes
%  - explore relations between hierarchy and rl
%  - develop strategies for the other game modes and observe performance
 % Conclusion, Discussion & Future work

%% ----------------------------------------------------------------
% Now begin the Appendices, including them as separate files

\addtocontents{toc}{\vspace{2em}} % Add a gap in the Contents, for aesthetics

\appendix % Cue to tell LaTeX that the following 'chapters' are Appendices

% Appendix A
\chapter{C \& H Domain}
\label{AppendixA}
\lhead{Appendix A. \emph{C \& H Domain}}

\setstretch{1.0}  % It is better to have smaller font and larger line spacing than the other way round
\lstinputlisting[language=Lisp, commentstyle=\color{green}, numbers=left, breaklines=true]{./Appendices/chdom.lisp}
\setstretch{1.3}  % It is better to have smaller font and larger line spacing than the other way round
	% C & H Domain

\addtocontents{toc}{\vspace{2em}}  % Add a gap in the Contents, for aesthetics
\backmatter

%% ----------------------------------------------------------------
\label{Bibliography}
\lhead{\emph{Bibliography}}  % Change the left side page header to "Bibliography"
\bibliographystyle{unsrtnat}  % Use the "unsrtnat" BibTeX style for formatting the Bibliography
\bibliography{Bibliography}  % The references (bibliography) information are stored in the file named "Bibliography.bib"

\end{document}  % The End
%% ----------------------------------------------------------------
