% Chapter 1
\chapter{Introduction} % Write in your own chapter title
\label{Chapter1}
\lhead{Chapter 1. \emph{Introduction}} % Write in your own chapter title to set the page header
The main goal of Artificial Intelligence (AI) in commercial video games is to
provide a rich gaming experience for the end user. Providing this experience
requires a careful balance between encoding prior knowledge, control by
developers and autonomous, learning AI. 

\section{Objective}
This thesis proposes a method that maintains a balance between encoding prior
knowledge, control by developers and autonomous, learning AI. The method
proposed is a novel combination of Hierarchical Task Network (HTN) planning and
Reinforcement Learning (RL). The HTN provides the structure to encode a variety
of strategies, whereas RL allows for learning the best strategy against a fixed
opponent. The method is empirically analyzed by applying it on the strategic
level of the video game called \emph{Killzone 3} (KZ3). KZ3 is a first-person
shooter (FPS), a genre of computer games where the player's viewing perspective
is shown through the eyes of their virtual character.

\section{Contributions}
In this thesis, we try to answer the following questions:
\begin{enumerate}
\item{Can we encode good strategies into a hierarchical task network?}
\item{Is it possible to learn the best strategy with respect to a fixed opponent?}
\item{Is it possible to adapt to a different fixed opponent?} 
\end{enumerate}
The first allows for a clear separation of domain language and problem solver
aswell as easy redesign of certain handcrafted strategies by the developer. The
second and third enable a richer, more personal gaming experience. By building
a learning mechanism on top of the HTN-Planner, the best (w.r.t. the current
fixed opponent) handcrafted strategy can be learned while keeping control over
the individual strategies, see chapter \ref{Chapter5} for the method proposed.

\section{Environment}
The environment in which the proposed method is implemented and tested is a
commercial first-person shooter called Killzone 3 developed by
\emph{Guerrilla}. KZ3 provides a good, real world test bed as its regarded as a
typical, high quality FPS game throughout the world. See chapter
\ref{Chapter4} for a detailed description.

\section{Scope}
In a typical FPS game there are several game modes, each with different
objectives that a team has to achieve to win the game. For the scope of this
thesis, the implementation is applied to the game mode called \emph{Capture and
Hold}.  However, it should be noted that the proposed method itself can be
applied to all existing game modes.\\
The objective of \emph{Capture and Hold} is to gain control over the key areas
on the map that can be captured by either team. To gain control over an area,
players must stand within the capture radius of the area and make sure that no
enemy soldiers are within capture range of that same area during the capture
process. If a team succeeds in keeping away the enemy long enough, the area
will change ownership to the capturing team. Once an area is captured the team
receives points for each time step the area is under their control. If the
maximum amount of points is gained (defined by the game) the team wins or,
alternatively, if the time limit is reached the team with the most points wins.
When both teams have the same amount of points the game results in a draw.

\section{Overview}
Chapter \ref{Chapter2} explains the background required for the proposed
method, chapter \ref{Chapter3} discusses related work from both the gaming
industry and academic world. In chapter \ref{Chapter4} the domain (Killzone 3)
is explained in detail and chapter \ref{Chapter5} introduces the proposed
method.  Experiments and results are presented in chapter \ref{Chapter6} and
finally the conclusions, discussion and future work are given in chapter
\ref{Chapter7}.
