% !TeX root = main.tex
% !TEX TS-program = pdflatex-pstricks

\usepackage{url}
\usepackage{amsmath}
\usepackage{amssymb}
\usepackage{amsfonts}
\usepackage{amsthm}
\usepackage{float}
\usepackage{natbib} 
\usepackage{fnwiaithesis}
\usepackage{txfonts}
\usepackage{mathtools}
\usepackage{multirow}
\usepackage[format=hang]{subfig}
\usepackage{caption}
\usepackage{centernot}
%\usepackage{MnSymbol}


\usepackage{graphicx} % support the \includegraphics command and options
\graphicspath{{./figures/}}
%\usepackage{epstopdf}

%nicer fonts??
\usepackage[T1]{fontenc}
%\usepackage{times}
%\usepackage{tgtermes}
\usepackage{lmodern}


%%% PAGE DIMENSIONS
\usepackage{geometry} % to change the page dimensions
\geometry{a4paper} % or letterpaper (US) or a5paper or....
% \geometry{margins=2in} % for example, change the margins to 2 inches all round
% \geometry{landscape} % set up the page for landscape
%   read geometry.pdf for detailed page layout information
%\setlength{\marginparwidth}{0pt}
%\setlength{\marginparsep}{0pt}
%\geometry{bindingoffset=1cm}

%%% PACKAGES
\usepackage{booktabs} % for much better looking tables
\usepackage{array} % for better arrays (eg matrices) in maths
\usepackage{paralist} % very flexible & customisable lists (eg. enumerate/itemize, etc.)
\usepackage{verbatim} % adds environment for commenting out blocks of text & for better verbatim
\usepackage{subfig} % make it possible to include more than one captioned figure/table in a single float
% These packages are all incorporated in the memoir class to one degree or another...

%%FLOATS
\usepackage{float}
%\usepackage{sidecap}
\floatstyle{boxed} 
\restylefloat{figure}


%%% ALGORITHMS
\usepackage[section]{algorithm}
%\usepackage[chapter]{algorithm}
\usepackage{algorithmic}

%%% HEADERS & FOOTERS
\usepackage{fancyhdr} % This should be set AFTER setting up the page geometry
\setlength{\headheight}{23pt}
\pagestyle{fancy} % options: empty , plain , fancy

\renewcommand{\chaptermark}[1]{\markboth{\thechapter.\ #1}{}}
\renewcommand{\sectionmark}[1]{\markright{\thesection.\ #1}}

% watch out with roman numerals remove all uppercases
\fancyhead[LE,RO]{\nouppercase{\rightmark}}
\fancyhead[LO,RE]{\nouppercase{\leftmark}}


%%% SECTION TITLE APPEARANCE
\usepackage{sectsty}
%\allsectionsfont{\sffamily\mdseries\upshape} % (See the fntguide.pdf for font help)
% (This matches ConTeXt defaults)

%%% ToC (table of contents) APPEARANCE
\usepackage[nottoc,notlof,notlot]{tocbibind} % Put the bibliography in the ToC
\usepackage[titles,subfigure]{tocloft} % Alter the style of the Table of Contents
\renewcommand{\cftsecfont}{\rmfamily\mdseries\upshape}
\renewcommand{\cftsecpagefont}{\rmfamily\mdseries\upshape} % No bold!
\usepackage{natbib}

%%WITH HYPERREF XELATEX
%\usepackage{pstricks}
%\usepackage[colorlinks=true, pdfborder={0 0 0}]{hyperref}

%%WITHOUT HYPERREF WITH PDFLATEX
\usepackage[pdf]{pstricks}
\usepackage{pst-plot}
\usepackage{pst-3dplot}
\usepackage{pstricks-add}

%%% END Article customizations
\usepackage{egameps}


\theoremstyle{plain}% default
\newtheorem{theorem}{Theorem}[section]
\newtheorem{lemma}[theorem]{Lemma}
\newtheorem{proposition}[theorem]{Proposition}
\newtheorem*{cor}{Corollary}


\theoremstyle{definition}
\newtheorem{definition}{Definition}[section]
\newtheorem{conj}{Conjecture}[section]
\newtheorem{exmp}{Example}[section]
\newtheorem{obs}{Observation}

\theoremstyle{remark}
\newtheorem*{remark}{Remark}
%\newtheorem*{note}{Note}
\newtheorem{case}{Case}

\newcommand{\tl}[1]{\href{C:/Users/Bart/Documents/Thesis/Library/#1}{#1}}
\newcommand{\tlb}[1]{\href{C:/Users/Bart/Documents/Thesis/Library/Books/#1}{#1}}

\newcommand{\real}[0]{\mathbb{R}}
\renewcommand{\natural}[0]{\mathbb{N}}
%\newcommand{\iff}[0]{\Leftrightarrow}

%algorithms
\renewcommand{\algorithmicrequire}{\textbf{Input:}}
\renewcommand{\algorithmicensure}{\textbf{Output:}}

%logics
\newcommand{\biglor}{\bigvee}
\newcommand{\meet}{\sqcap}
\newcommand{\norm}[1]{{\left\|{#1}\right\|}}
\newcommand{\Tau}{\mathrm{T}}
%\newcommand{\Alpha}{\mathrm{A}}
\newcommand{\Alpha}{\mathcal{V}}

%Maths
\DeclareMathOperator*{\region}{region}
\DeclareMathOperator*{\argmax}{argmax}
\DeclareMathOperator*{\argmin}{argmin}
\DeclareMathOperator*{\cav}{cav}
\DeclareMathOperator*{\vex}{vex}
\DeclareMathOperator*{\rnf}{rnf}
\DeclareMathOperator*{\ch}{conv}
\DeclareMathOperator*{\ext}{ext}
\DeclareMathOperator*{\V2H}{V2H}
\DeclareMathOperator*{\H2V}{H2V}
%partition =  mathcal
%class in partition = mathfrak
%set = capital
%element in set is lowercase.
% capial can also be 1,2,3,...,K
% K, L are usual counting variables
% counting super script
% identifiers subscript
% capital T is transpose (or time)
% e_d^D, unit vector in D space