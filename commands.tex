\newcommand{\code}[1]{\texttt{#1}}

\newcommand{\fig}[3]{
	\begin{figure}[!ht]
	\centering
	\includegraphics[scale=#3]{img/#1}
	%\includegraphics[width=420px]{img/#1}
	\caption{#2}
	\label{fig:#1}
	\end{figure}
}
\newcommand{\fignocaption}[3]{
	\begin{figure}[!ht]
	\centering
	\includegraphics[scale=#3]{img/#1}
	\label{fig:#1}
	\end{figure}
}

\newcommand{\figw}[3]{
	\begin{figure}[!ht]
	\centering
	\includegraphics[width=#3]{img/#1}
	\caption{#2}
	\label{fig:#1}
	\end{figure}
}

\newcommand{\figs}[6]{
	\begin{figure}[!ht]
	\centering
	\subfigure[#5]{
		% width=14cm
		\includegraphics[width=14cm]{img/#2}
		\label{fig:#2}
	}
	\subfigure[#6]{
		% width=14cm
		\includegraphics[width=14cm]{img/#3}
		\label{fig:#3}
	}
	\caption{#4}
	\label{fig:#1}
	\end{figure}
}
\newcommand{\figsSmall}[6]{
	\begin{figure}[!ht]
	\centering
	\subfigure[#5]{
		% width=14cm
		\includegraphics[width=5cm]{img/#2}
		\label{fig:#2}
	}
	\subfigure[#6]{
		% width=14cm
		\includegraphics[width=5cm]{img/#3}
		\label{fig:#3}
	}
	\caption{#4}
	\label{fig:#1}
	\end{figure}
}
