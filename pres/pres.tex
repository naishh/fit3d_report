\newcommand{\code}[1]{\texttt{#1}}

\newcommand{\fig}[3]{
	\begin{figure}[!ht]
	\centering
	\includegraphics[scale=#3]{img/#1}
	%\includegraphics[width=420px]{img/#1}
	\caption{#2}
	\label{fig:#1}
	\end{figure}
}
\newcommand{\fignocaption}[3]{
	\begin{figure}[!ht]
	\centering
	\includegraphics[scale=#3]{img/#1}
	\label{fig:#1}
	\end{figure}
}

\newcommand{\figw}[3]{
	\begin{figure}[!ht]
	\centering
	\includegraphics[width=#3]{img/#1}
	\caption{#2}
	\label{fig:#1}
	\end{figure}
}

\newcommand{\figs}[6]{
	\begin{figure}[!ht]
	\centering
	\subfigure[#5]{
		\includegraphics[width=14cm]{img/#2}
		\label{fig:#2}
	}
	\subfigure[#6]{
		\includegraphics[width=14cm]{img/#3}
		\label{fig:#3}
	}
	\caption{#4}
	\label{fig:#1}
	\end{figure}
}
\newcommand{\figsSmall}[6]{
	\begin{figure}[!ht]
	\centering
	\subfigure[#5]{
		% width=14cm
		\includegraphics[width=5cm]{img/#2}
		\label{fig:#2}
	}
	\subfigure[#6]{
		% width=14cm
		\includegraphics[width=5cm]{img/#3}
		\label{fig:#3}
	}
	\caption{#4}
	\label{fig:#1}
	\end{figure}
}

%todo beamer package includen
%todo documentclass?

\title{Welcome}
%todo opmaak

{\LARGE \sc{Semantic annotation of urban scenes:}\\Skyline and window detection}

Speaker: Tjerk Kostelijk
Supervisors:
Isaac Esteban
Prof. dr ir Frans C. A. Groen

%Committee:
%Prof. dr ir Frans C. A. Groen
%dr. P.H. Rodenburg
%dr. Arnoud Visser

July 6th, 2012


\section{Introduction}
\title{Experiment}
% blank dia
%notes:
%this is for a little warming up
%and to answer the question why I did my research


\title{Experiment}
%todo 1 sec
\fig{floriande_back.eps}{420px}


\title{Did you see?}
%todo Yes en No op bord schrijven
\begin{itemize}
	\item building
	\item tree
	\item bicycle
	%todo spellcheck
	\item street lamb
	\item red car
	\item blue car
	\item brand of the car?
	% blank dia
\end{itemize}


\title{Experiment}
% blank dia
\title{Experiment}
\fig{floriande_back.eps}{420px}


\title{Q}
\item Why are we so good at object/depth recognition?
\item How can we apply vision to a computer system?
	\begin{itemize}
		\item {\textit{Computer Vision}}
		% this is the domain which interest me the most and on which I did my research
	\end{itemize}
%These are two questions that keeps us AI people bussy
% I will answer the first one shortley and the rest of the presentation is about
% the second question


\title{Outline}
%todo print toc
% -human perception
% -skyline detection
% -3d building reconstruction
% -window detection
 
%todo think of transition

\title{Human perception}
% lets talk about how you perceived the scene of the building
\item depth cue, binocular disparity
% vingertest
% everybody raise your index finger, 
% now close your left eye and switch to see your finger hopping
% do the same with your finger further away
% the displacement is smaller

%We use two eyes and look at the sam scene from slightly diffeent angles
%We perceive two different images, if an object appears close the differenc
%displacement difference between the images is high
%this makes it possible to triangulate the distance to an abject with a high
%degree of accuracy

\begin{itemize}
\item Classify objects: feature detection
	\begin{itemize}
		\item straight lines
		\item right angles 
	\end{itemize}
	%todo image of H and L 
\end{itemize}


\title{Where is my research about?}
\item{Annotation of urban scenes}
	\begin{itemize}
		\item Skyline detection
			\fig{outputSkylineSpil-Im29.eps}{width=200px}
		\item 3D building reconstruction
			% todo plaatje floriande -> 3d model
			\fig{3dModel}{width=200px}
		\item Window detection
	\end{itemize}

%todo insert TOC at every section
\section{Skylinedetection}
\title{Application examples of annotation of urban scenes}
%What can we do with this annotation of urban scenes?}
\begin{itemize}
	\item 3D city models
	%todo image
	%todo paper downloaden met dat 3d city model plaatje, auters: muller en wonka
	\item Driving simulation
	\item Augmented reality
	%todo image
	\item Building recognition
	\item Analysis building deformation
	\begin{itemize}
		% todo foto van de detectie dingen van noord-zuidlijn
		\item 	'noord-zuidlijn'
	\end{itemize}
\end{itemize}


\title{Skyline detection application example}
\begin{itemize}
	\item Horizon detection for Unmanned Air Vehicles
	%notes:used to stabilize the vehicle
	% todo vet plaatje op google vinden van een uav
	\fig{p_uav_skyline_all.eps}{width=420px}
\end{itemize}

\title{Skyline in urban scenes}% the method
%we use it to detect the contour of the building
\begin{itemize}
	\item Canny edge detection 
	\item Result: Binary image (edge or no edge)
	% evt computerprogrammatjuuuu
		%edge represents a high change in color 
	%\item Pre-precessing: Gaussian smoothing
	\item top sharp edge assumption\\
	\emph{"The first sharp edge (seen from top to bottom) in the image represents the skyline."}
\end{itemize}


\title{Skyline detection algorithm}% the method
\begin{itemize}
	\item The image is sliced in #w pixelcolumns
	\item Each column present #h binary edge values (edge or no edge)
	\item y-location of the first edge value is stored 
\end{itemize}



\title{Skyline detection Result}% the method
\fig{e_floriande_canny_050.eps}

\subsection{Future research}
\title{Hypothesis based skyline detection, assumption}
\emph{"The first sharp edge (seen from top to bottom) in the image represents the skyline."}
\item Example of a scene where this assumption is violated
\fig{outputSkylineSpil-Im13-thresh030.eps}

\title{Hypothesis based skyline detection, assumption}
\item Change of assumption
\emph{"The skyline is part of the first $n$ sharp edges (seen from top) (e.g.
$n=3$)}
\fig{folkert_edgeHypothesis.eps}

\title{Hypothesis based skyline detection, algorithm}
\begin{itemize}
\item Use exsisting column based approach
\item Generate $n$ hypothesis
\item classify hypothesis with additional info
	\begin{itemize}
	\item texture 
	\item color
	\item height variation 
	\end{itemize}
\end{itemize}

\title{Expected result on hypothesis classification based on color}
\fig{outputSkylineSpil-Im13-thresh070.eps}

\title{Expected result on hypothesis classification based on heigth variation}
\fig{outputSkylineIm3-3-zoom.eps}
\fig{outputSkylineIm3-3-zoom-con.eps}

%todo close up skyline above sky below bricks
%todo close up skyline above sky below sky (lamp) 


%notes
%The existing column based approach could be used to generate $n$ hypotheses and we can build
%an algorithm on top of that to classify which hypothesis is right (i.e. which edge
%presents the skyline).
%
%The hypothesis classifier must gather additional information. We could
%discriminate the hypothesis for example by texture, color distributions or height
%variation. We discuss the last two.


\section{Extracting the 3D model}
\title{Overview}
%notes a skyline detection application
\begin{itemize}
\item Create (top view) 2D model of the scene using \emph{Openstreetmap}
\fig{openstreetmap}
\item Align 2D model with the scene using \emph{FIT3D toolbox}
\item Transform the 2D model to a 3D model by extending the walls
\end{itemize}

\title{Extract 2D model}
\fig{openstreetmap}
% whiteboard draw 2d model
%-----------------------------------------------------

\title{Align 2D model}
\begin{itemize}
\item \emph{FIT3D toolbox}
% out of the scope, is explained in my thesis
	\begin{itemize}
	\item Input: sequence of images that contain different views of the building
	\item Output: a 3D point cloud of the building
	\end{itemize}
\figsHor{outputSkylineIm3-3.eps}{pc_3d.eps} %{ 3D point cloud of the walls of the building}{0.5}
\end{itemize}


\title{Align 2D model}
\fig{pc_topview.eps} %{The projected 3D point cloud of the walls of the


\title{Align 2D model}
\fig{pc_topview_linefit.eps} %{M2, The fitted line segments define (a top view of
%---------------- notes ------------------------------
% explain point to point correspondence 
%-----------------------------------------------------


\title{Align 2D model}
\fig{pc_topview2DModel.eps} %{The 2D model M1 aligned with the fitted line

\title{Results}
\fig{3dModel} %{The basic 3D model, generated by extending the 

\title{Wall height estimation}
% todo use this as a motivation for skyline detection
\begin{itemize}
%---------------- notes ------------------------------
% because 3D model based on aerial images (openstreetmap)
% -> Wall heights of 3D model are not accurate
% explain point to point correspondence 
% although it looks like a nice 3D model the wall heights..
%-----------------------------------------------------
%
\item Wall heights of 3D model are not accurate
\figsHor{3dModel}{floriande_back.eps} %{The walls of the building differ in height: the back
\item Improve the 3D model by wall height estimation
\end{itemize}

\title{Assumptions}
%---------------- notes ------------------------------
% Because we want to estimate the height of the building walls of the 3D
% model, we need to know how high the walls in the 2D images are. 
%---------------- notes ------------------------------
\emph{Straight lines in the skyline are likely to come from the building contour}
\emph{Flat roof assumption -> building contour is equal to upper side of building walls}

\title{Extracting line segments}
% a widely used method for extracting line segments is the Hough transform
\item Output of skyline detector
%todo insert binary skyline image
\item Hough transform
%---------------- notes ------------------------------
% the technique behind this method is discussed in detail in my thesis
% unfortunatelly, for now we don't have enough time to dive in to this topic
%---------------- notes ------------------------------

%\item Basic idea: Lines -> points in Hough space
%\item Straight line segment that spans $y=mx+b$ represents $(m,b)$ point in Hough parameter space

%---------------- notes ------------------------------
% Because the algorithm detects straight lines containing only skyline pixels
% it returns only the straight parts of the skyline.
%---------------- notes ------------------------------
\fig{outputHoughlines2d} %{Three best ranked lines of the Hough transform on the skyline detector match the three most prominent displayed building walls}{0.4}


\title{Project to the 3D model}
\begin{itemize}
\item Estimate wall heights 
\item project line segments to specific walls of 3D model
\fig{coordinateSystemsCopy.eps} %{The blue line spanned by the camera center
%---------------- notes ------------------------------
% From image point (2D) to possible points in scene (3D)} 
%---------------- notes ------------------------------
\item
\end{itemize}


\title{Results 1/3}
%\fignocaption{overlaytypes}{}{0.37}
\fig{skyline_proj.eps} %{The Houghlines collected from different views projected on their corresponding


\title{Results 2/3}
\fig{outputHoughlines3d} %{The Houghlines collected from different views re-projected on the 3D model according to the line-wall correspondence.}{0.5}


\title{Results 3/3}
\fig{outputMutateBuilding}{Improved 3D model}{0.6}

\title{Future research}
%todo als tijd over

%\fig{typesOfRoofs}{Different types of roofs}{0.4}
%\fig{gableRoofWood.eps}{The parts of the 3D model for the Gable roof}{0.6}

\title{Overview}
%todo highlight window detection

\title{Outline Window detection}
\begin{itemize}
\item Method I: Connected corner approach 
	\begin{itemize}
	\item invariant to viewing direction
	\end{itemize}
\item Facade rectification 
\item Method II: Histogram based approach
	\begin{itemize}
	\item rectified facades 
	\item uses Histograms of Houghlines
	\end{itemize}
\end{itemize}
% this wil come clear when I explain it 


%---------------- notes ------------------------------
recall that this thesis was about annotation of urban scenes
we annotated the 3D structure of the scene
next topic is window detection
%---------------- notes ------------------------------





\section{Interactive demo}
osgviewer demo of 3d point cloud and images
demo of matlab progje



