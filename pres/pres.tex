% PACKAGES
\usepackage{epsfig}
\usepackage{graphicx, subfigure}
\usepackage{float}

\usepackage{amsmath} % for the argmin
\usepackage{amsfonts}
\usepackage{url}

% COMMANDS

\newcommand{\fig}[2]{
	\begin{figure}[!ht]
	\centering
	%\includegraphics[=#3]{../img/#1}
	\includegraphics[width=#2]{../img/#1}
	%\caption{#2}
	\label{fig:#1}
	\end{figure}
}
\newcommand{\fignocaption}[3]{
	\begin{figure}[!ht]
	\centering
	\includegraphics[scale=#3]{img/#1}
	\label{fig:#1}
	\end{figure}
}


%todo beamer package includen
%todo documentclass?

\section{Welcome}
%todo opmaak

{\LARGE \sc{Semantic annotation of urban scenes:}\\Skyline and window detection}

Speaker: Tjerk Kostelijk
Supervisors:
Isaac Esteban
Prof. dr ir Frans C. A. Groen

%Committee:
%Prof. dr ir Frans C. A. Groen
%dr. P.H. Rodenburg
%dr. Arnoud Visser

July 6th, 2012

\section{Experiment}
% blank dia
%notes:
%this is for a little warming up
%and to answer the question why I did my research


\section{Experiment}
%todo 1 sec
\fig{floriande_back.eps}{420px}


\section{Did you see?}
%todo Yes en No op bord schrijven
\begin{itemize}
	\item building
	\item tree
	\item bicycle
	%todo spellcheck
	\item street lamb
	\item red car
	\item blue car
	\item brand of the car?
	% blank dia
\end{itemize}


\section{Experiment}
% blank dia
\section{Experiment}
\fig{floriande_back.eps}{420px}


\section{Q}
\item Why are we so good at object/depth recognition?
\item How can we apply vision to a computer system?
	\begin{itemize}
		\item {\textit{Computer Vision}}
		% this is the domain which interest me the most and on which I did my research
	\end{itemize}
%These are two questions that keeps us AI people bussy
% I will answer the first one shortley and the rest of the presentation is about
% the second question


\section{Outline}
%todo print toc
% -human perception
% -skyline detection
% -3d building reconstruction
% -window detection
 
%todo think of transition

\section{Human perception}
% lets talk about how you perceived the scene of the building
\item depth cue, binocular disparity
% vingertest
% everybody raise your index finger, 
% now close your left eye and switch to see your finger hopping
% do the same with your finger further away
% the displacement is smaller

%We use two eyes and look at the sam scene from slightly diffeent angles
%We perceive two different images, if an object appears close the differenc
%displacement difference between the images is high
%this makes it possible to triangulate the distance to an abject with a high
%degree of accuracy

\begin{itemize}
\item Classify objects: feature detection
	\begin{itemize}
		\item straight lines
		\item right angles 
	\end{itemize}
	%todo image of H and L 
\end{itemize}


\section{Where is my research about?}
\item{Annotation of urban scenes}
	\begin{itemize}
		\item Skyline detection
			\fig{outputSkylineSpil-Im29.eps}{width=200px}
		\item 3D building reconstruction
			% todo plaatje floriande -> 3d model
			\fig{3dModel}{width=200px}
		\item Window detection
	\end{itemize}

\section{Application examples of annotation of urban scenes}
%What can we do with this annotation of urban scenes?}
\begin{itemize}
	\item 3D city models
	%todo image
	%todo paper downloaden met dat 3d city model plaatje, auters: muller en wonka
	\item Driving simulation
	\item Augmented reality
	%todo image
	\item Building recognition
	\item Analysis building deformation
	\begin{itemize}
		% todo foto van de detectie dingen van noord-zuidlijn
		\item 	'noord-zuidlijn'
	\end{itemize}
\end{itemize}


\section{Skyline detection application example}
\begin{itemize}
	\item Horizon detection for Unmanned Air Vehicles
	%notes:used to stabilize the vehicle
	% todo vet plaatje op google vinden van een uav
	\fig{p_uav_skyline_all.eps}{width=420px}
\end{itemize}

\section{Skyline in urban scenes}% the method
%we use it to detect the contour of the building
\begin{itemize}
	\item Canny edge detection 
	\item Result: Binary image (edge or no edge)
	% evt computerprogrammatjuuuu
		%edge represents a high change in color 
	%\item Pre-precessing: Gaussian smoothing
	\item top sharp edge assumption\\
	\emph{"The first sharp edge (seen from top to bottom) in the image represents the skyline."}
\end{itemize}


\section{Skyline detection algorithm}% the method
\begin{itemize}
	\item The image is sliced in #w pixelcolumns
	\item Each column present #h binary edge values (edge or no edge)
	\item y-location of the first edge value is stored 
\end{itemize}



\subsection{Skyline detection Result}% the method
\fig{e_floriande_canny_050.eps}

\subsection{Future research}
\title{Hypothesis based skyline detection}
\emph{"The first sharp edge (seen from top to bottom) in the image represents the skyline."}
\item Example of a scene where this assumption is violated
\fig{folkert_edgeHypothesis.eps}
\item 



