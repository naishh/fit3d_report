%todo opmaak
\documentclass{beamer}
\usetheme[compress]{Dresden}
\setbeamertemplate{navigation symbols}{} 
% PACKAGES
\usepackage{epsfig}
\usepackage{graphicx, subfigure}
\usepackage{float}

\usepackage{amsmath} % for the argmin
\usepackage{amsfonts}
\usepackage{url}

% COMMANDS

\newcommand{\fig}[2]{
	\begin{figure}[!ht]
	\centering
	%\includegraphics[=#3]{../img/#1}
	\includegraphics[width=#2]{../img/#1}
	%\caption{#2}
	\label{fig:#1}
	\end{figure}
}
\newcommand{\fignocaption}[3]{
	\begin{figure}[!ht]
	\centering
	\includegraphics[scale=#3]{img/#1}
	\label{fig:#1}
	\end{figure}
}




%{\LARGE \sc{Semantic annotation of urban scenes:}\\Skyline and window detection}

% notes: say welcome to my master thesis' defence
\title{\LARGE \sc{Semantic annotation of urban scenes:}\\Skyline and window detection}
\subtitle{Speaker: Tjerk Kostelijk}
\author{Supervisors: \\Isaac Esteban\\Prof. dr ir Frans C. A. Groen}
\date{\today}
\begin{document}


%Committee:
%Prof. dr ir Frans C. A. Groen
%dr. P.H. Rodenburg
%dr. Arnoud Visser


\frame{\titlepage}

%\section[Outline]{}
%\frame{\tableofcontents}

% outline
\AtBeginSection[]
{
 \begin{frame}
  \frametitle{Outline}
  \small
  \tableofcontents[currentsection,hideothersubsections]
  \normalsize
 \end{frame}
}


\section{Introduction}
	%todo subsections

%-------------------------------------------------------------------
\frame
{
	\frametitle{Experiment}
	% blank dia
	%notes:
	%this is for a little warming up
	%and to answer the question why I did my research
}


%-------------------------------------------------------------------
\frame
{
	\frametitle{Experiment}
	%todo 1 sec
	\fig{floriande_back.eps}{420px}
}


%-------------------------------------------------------------------
\frame
{
	\frametitle{Did you see?}
	%todo Yes en No op bord schrijven
	\begin{itemize}
		\item  <+-| alert@+> building
		\item  <+-| alert@+> tree
		\item  <+-| alert@+> bicycle
		%todo spellcheck
		\item  <+-| alert@+> street lamb
		\item  <+-| alert@+> red car
		\item  <+-| alert@+> blue car
		\item  <+-| alert@+> brand of the car?
		% blank dia
	\end{itemize}
}


%-------------------------------------------------------------------
\frame
{
	\frametitle{Experiment}
	% blank dia
}

%-------------------------------------------------------------------
\frame
{
	\frametitle{Experiment}
	\fig{floriande_back.eps}{420px}
}


%-------------------------------------------------------------------
\frame
{
	\frametitle{Q}
	\item  <+-| alert@+> Why are we so good at object/depth recognition?
	\item  <+-| alert@+> How can we apply vision to a computer system?
		\begin{itemize}
			\item  <+-| alert@+> {\textit{Computer Vision}}
			% this is the domain which interest me the most and on which I did my research
		\end{itemize}
	%These are two questions that keeps us AI people bussy
	% I will answer the first one shortley and the rest of the presentation is about
	% the second question
}


%-------------------------------------------------------------------
\frame
{
	\frametitle{Outline}
	%todo print toc
	% -human perception
	% -skyline detection
	% -3d building reconstruction
	% -window detection
	 
	%todo think of transition
}

%-------------------------------------------------------------------
\frame
{
	\frametitle{Human perception}
	% lets talk about how you perceived the scene of the building
	\item  <+-| alert@+> depth cue, binocular disparity
	% vingertest
	% everybody raise your index finger, 
	% now close your left eye and switch to see your finger hopping
	% do the same with your finger further away
	% the displacement is smaller

	%We use two eyes and look at the sam scene from slightly diffeent angles
	%We perceive two different images, if an object appears close the differenc
	%displacement difference between the images is high
	%this makes it possible to triangulate the distance to an abject with a high
	%degree of accuracy

	\begin{itemize}
	\item  <+-| alert@+> Classify objects: feature detection
		\begin{itemize}
			\item  <+-| alert@+> straight lines
			\item  <+-| alert@+> right angles 
		\end{itemize}
		%todo image of H and L 
	\end{itemize}
}


%-------------------------------------------------------------------
\frame
{
	\frametitle{Where is my research about?}
	\item  <+-| alert@+>{Annotation of urban scenes}
		\begin{itemize}
			\item  <+-| alert@+> Skyline detection
				\fig{outputSkylineSpil-Im29.eps}{width=200px}
			\item  <+-| alert@+> 3D building reconstruction
				% todo plaatje floriande -> 3d model
				\fig{3dModel}{width=200px}
			\item  <+-| alert@+> Window detection
		\end{itemize}

	%todo insert TOC at every section
}

\section{Skylinedetection}
%-------------------------------------------------------------------
\frame
{
	\frametitle{Application examples of annotation of urban scenes}
	%What can we do with this annotation of urban scenes?}
	\begin{itemize}
		\item  <+-| alert@+> 3D city models
		%todo image
		%todo paper downloaden met dat 3d city model plaatje, auters: muller en wonka
		\item  <+-| alert@+> Driving simulation
		\item  <+-| alert@+> Augmented reality
		%todo image
		\item  <+-| alert@+> Building recognition
		\item  <+-| alert@+> Analysis building deformation
		\begin{itemize}
			% todo foto van de detectie dingen van noord-zuidlijn
			\item  <+-| alert@+> 	'noord-zuidlijn'
		\end{itemize}
	\end{itemize}
}


%-------------------------------------------------------------------
\frame
{
	\frametitle{Skyline detection application example}
	\begin{itemize}
		\item  <+-| alert@+> Horizon detection for Unmanned Air Vehicles
		%notes:used to stabilize the vehicle
		% todo vet plaatje op google vinden van een uav
		\fig{p_uav_skyline_all.eps}{width=420px}
	\end{itemize}
}

%-------------------------------------------------------------------
\frame
{
	\frametitle{Skyline in urban scenes}% the method
	%we use it to detect the contour of the building
	\begin{itemize}
		\item  <+-| alert@+> Canny edge detection 
		\item  <+-| alert@+> Result: Binary image (edge or no edge)
		% evt computerprogrammatjuuuu
			%edge represents a high change in color 
		%\item  <+-| alert@+> Pre-precessing: Gaussian smoothing
		\item  <+-| alert@+> top sharp edge assumption\\
		\emph{"The first sharp edge (seen from top to bottom) in the image represents the skyline."}
	\end{itemize}
}


%-------------------------------------------------------------------
\frame
{
	\frametitle{Skyline detection algorithm}% the method
	\begin{itemize}
		\item  <+-| alert@+> The image is sliced in #w pixelcolumns
		\item  <+-| alert@+> Each column present #h binary edge values (edge or no edge)
		\item  <+-| alert@+> y-location of the first edge value is stored 
	\end{itemize}
}



%-------------------------------------------------------------------
\frame
{
	\frametitle{Skyline detection Result}% the method
	\fig{e_floriande_canny_050.eps}
}

\subsection{Future research}
%-------------------------------------------------------------------
\frame
{
	\frametitle{Hypothesis based skyline detection, assumption}
	\emph{"The first sharp edge (seen from top to bottom) in the image represents the skyline."}
	\item  <+-| alert@+> Example of a scene where this assumption is violated
	\fig{outputSkylineSpil-Im13-thresh030.eps}
}

%-------------------------------------------------------------------
\frame
{
	\frametitle{Hypothesis based skyline detection, assumption}
	\item  <+-| alert@+> Change of assumption
	\emph{"The skyline is part of the first $n$ sharp edges (seen from top) (e.g.
	$n=3$)}
	\fig{folkert_edgeHypothesis.eps}
}

%-------------------------------------------------------------------
\frame
{
	\frametitle{Hypothesis based skyline detection, algorithm}
	\begin{itemize}
	\item  <+-| alert@+> Use exsisting column based approach
	\item  <+-| alert@+> Generate $n$ hypothesis
	\item  <+-| alert@+> classify hypothesis with additional info
		\begin{itemize}
		\item  <+-| alert@+> texture 
		\item  <+-| alert@+> color
		\item  <+-| alert@+> height variation 
		\end{itemize}
	\end{itemize}
}

%-------------------------------------------------------------------
\frame
{
	\frametitle{Expected result on hypothesis classification based on color}
	\fig{outputSkylineSpil-Im13-thresh070.eps}
}

%-------------------------------------------------------------------
\frame
{
	\frametitle{Expected result on hypothesis classification based on heigth variation}
	\fig{outputSkylineIm3-3-zoom.eps}
	\fig{outputSkylineIm3-3-zoom-con.eps}
}

	%todo close up skyline above sky below bricks
	%todo close up skyline above sky below sky (lamp) 


	%notes
	%The existing column based approach could be used to generate $n$ hypotheses and we can build
	%an algorithm on top of that to classify which hypothesis is right (i.e. which edge
	%presents the skyline).
	%
	%The hypothesis classifier must gather additional information. We could
	%discriminate the hypothesis for example by texture, color distributions or height
	%variation. We discuss the last two.


\section{Extracting the 3D model}
%-------------------------------------------------------------------
\frame
{
	\frametitle{Overview}
	%notes a skyline detection application
	\begin{itemize}
	\item  <+-| alert@+> Create (top view) 2D model of the scene using \emph{Openstreetmap}
	\fig{openstreetmap}
	\item  <+-| alert@+> Align 2D model with the scene using \emph{FIT3D toolbox}
	\item  <+-| alert@+> Transform the 2D model to a 3D model by extending the walls
	\end{itemize}
}

%-------------------------------------------------------------------
\frame
{
	\frametitle{Extract 2D model}
	\fig{openstreetmap}
	%--------notes----------------------------------------
	% whiteboard draw 2d model
	%-----------------------------------------------------
}

%-------------------------------------------------------------------
\frame
{
	\frametitle{Align 2D model}
	\begin{itemize}
	\item  <+-| alert@+> \emph{FIT3D toolbox}
	% out of the scope, is explained in my thesis
		\begin{itemize}
		\item  <+-| alert@+> Input: sequence of images that contain different views of the building
		\item  <+-| alert@+> Output: a 3D point cloud of the building
		\end{itemize}
	\figsHor{outputSkylineIm3-3.eps}{pc_3d.eps} %{ 3D point cloud of the walls of the building}{0.5}
	\end{itemize}
}


%-------------------------------------------------------------------
\frame
{
	\frametitle{Align 2D model}
	\fig{pc_topview.eps} %{The projected 3D point cloud of the walls of the
}


%-------------------------------------------------------------------
\frame
{
	\frametitle{Align 2D model}
	\fig{pc_topview_linefit.eps} %{M2, The fitted line segments define (a top view of
	%---------------- notes ------------------------------
	% explain point to point correspondence 
	%-----------------------------------------------------
}


%-------------------------------------------------------------------
\frame
{
	\frametitle{Align 2D model}
	\fig{pc_topview2DModel.eps} %{The 2D model M1 aligned with the fitted line
}

%-------------------------------------------------------------------
\frame
{
	\frametitle{Results}
	\fig{3dModel} %{The basic 3D model, generated by extending the 
}

%-------------------------------------------------------------------
\frame
{
	\frametitle{Wall height estimation}
	% todo use this as a motivation for skyline detection
	\begin{itemize}
	%---------------- notes ------------------------------
	% because 3D model based on aerial images (openstreetmap)
	% -> Wall heights of 3D model are not accurate
	% explain point to point correspondence 
	% although it looks like a nice 3D model the wall heights..
	%-----------------------------------------------------
	%
	\item  <+-| alert@+> Wall heights of 3D model are not accurate
	\figsHor{3dModel}{floriande_back.eps} %{The walls of the building differ in height: the back
	\item  <+-| alert@+> Improve the 3D model by wall height estimation
	\end{itemize}
}

%-------------------------------------------------------------------
\frame
{
	\frametitle{Assumptions}
	%---------------- notes ------------------------------
	% Because we want to estimate the height of the building walls of the 3D
	% model, we need to know how high the walls in the 2D images are. 
	%---------------- notes ------------------------------
	\emph{Straight lines in the skyline are likely to come from the building contour}
	\emph{Flat roof assumption -> building contour is equal to upper side of building walls}
}

%-------------------------------------------------------------------
\frame
{
	\frametitle{Extracting line segments}
	% a widely used method for extracting line segments is the Hough transform
	\item  <+-| alert@+> Output of skyline detector
	%todo insert binary skyline image
	\item  <+-| alert@+> Hough transform
	%---------------- notes ------------------------------
	% the technique behind this method is discussed in detail in my thesis
	% unfortunatelly, for now we don't have enough time to dive in to this topic
	%---------------- notes ------------------------------

	%\item  <+-| alert@+> Basic idea: Lines -> points in Hough space
	%\item  <+-| alert@+> Straight line segment that spans $y=mx+b$ represents $(m,b)$ point in Hough parameter space

	%---------------- notes ------------------------------
	% Because the algorithm detects straight lines containing only skyline pixels
	% it returns only the straight parts of the skyline.
	%---------------- notes ------------------------------
	\fig{outputHoughlines2d} %{Three best ranked lines of the Hough transform on the skyline detector match the three most prominent displayed building walls}{0.4}
}


%-------------------------------------------------------------------
\frame
{
	\frametitle{Project to the 3D model}
	\begin{itemize}
	\item  <+-| alert@+> Estimate wall heights 
	\item  <+-| alert@+> project line segments to specific walls of 3D model
	\fig{coordinateSystemsCopy.eps} %{The blue line spanned by the camera center
	%---------------- notes ------------------------------
	% From image point (2D) to possible points in scene (3D)} 
	%---------------- notes ------------------------------
	\item  <+-| alert@+>
	\end{itemize}
}


%-------------------------------------------------------------------
\frame
{
	\frametitle{Results 1/3}
	%\fignocaption{overlaytypes}{}{0.37}
	\fig{skyline_proj.eps} %{The Houghlines collected from different views projected on their corresponding
}


%-------------------------------------------------------------------
\frame
{
	\frametitle{Results 2/3}
	\fig{outputHoughlines3d} %{The Houghlines collected from different views re-projected on the 3D model according to the line-wall correspondence.}{0.5}
}


%-------------------------------------------------------------------
\frame
{
	\frametitle{Results 3/3}
	\fig{outputMutateBuilding}{Improved 3D model}{0.6}
}

%-------------------------------------------------------------------
\frame
{
	\frametitle{Future research}
	%todo als tijd over

	%\fig{typesOfRoofs}{Different types of roofs}{0.4}
	%\fig{gableRoofWood.eps}{The parts of the 3D model for the Gable roof}{0.6}
}

%-------------------------------------------------------------------
\frame
{
	\frametitle{Overview}
	%todo highlight window detection
}




%-------------------------------------------------------------------
\frame
{
	\frametitle{Outline Window detection}
	%---------------- notes ------------------------------
	%recall that this thesis was about annotation of urban scenes
	%we annotated the 3D structure of the scene
	%next topic is window detection
	%---------------- notes ------------------------------
	\begin{itemize}
	\item  <+-| alert@+> Method I: Connected corner approach 
		\begin{itemize}
		\item  <+-| alert@+> invariant to viewing direction
		\end{itemize}
	\item  <+-| alert@+> Facade rectification 
	\item  <+-| alert@+> Method II: Histogram based approach
		\begin{itemize}
		\item  <+-| alert@+> rectified facades 
		\item  <+-| alert@+> uses Histograms of Houghlines
		\end{itemize}
	\end{itemize}
	% this will become clear when I explain it 
}

%-------------------------------------------------------------------
\frame
{
	\frametitle{Method I: Connected corner approach}
}


%-------------------------------------------------------------------
\frame
{
	\frametitle{Situation}
	\begin{itemize}
	%---------------- notes ------------------------------
	%notes Window contains subwindows enclosed in rectangular window frames 
	%As the color of the window frames differ from the glass, the amount of
	%horizontal and vertical edges is large at these locations.  
	%Some horizontal and vertical edges come from the same window frame, therefore they often
	%---------------- notes ------------------------------
	\item  <+-| alert@+> Window frames 
	\item  <+-| alert@+> Difference color window frame, color glass
	\item  <+-| alert@+> Produce edges in horizontal and vertical direction
	\end{itemize}
}


%-------------------------------------------------------------------
\frame
{
	\frametitle{The idea}
	\item  <+-| alert@+> Connected corner}
	\item  <+-| alert@+> \emph{"horizontal and vertical edges that come from same (sub)window frame share
	a corner"}
}


%-------------------------------------------------------------------
\frame
{
	\frametitle{Edge detection and Houghline extraction}
	%---------------- notes ------------------------------
	%edge detection -> straight lines 
	%---------------- notes ------------------------------
	%todo
}

%	\subsubsection{Connected corners extraction}
%
%	\subsubsection{Window area extraction}
%
%	\subsubsection{Results}
%
%	\subsubsection{Future research} %\subsubsection{Method I: Connected corner approach} 
%
%	\subsection{Facade rectification}
%
%	\subsubsection{Introduction}
%
%	\subsubsection{3D plane based rectification} 
%
%	\subsubsection{Results} % facade rectification
%
%	\subsection{Datasets}
%
%	\subsubsection{Anne1}
%
%	\subsubsection{Anne2}
%
%	\subsubsection{Dirk}
%
%	\subsection{Method II: Histogram based approach} 
%
%	\subsubsection{Introduction}
%
%	\subsubsection{Extracting the window alignment}
%
%	\subsubsection{Basic window classification (based on line amount)}
%
%	\subsubsection{Improved window classification (based on shape of the histogram function)}
%
%	%\subsubsection{Comparison of basic and improved window classification}
%
%	\subsection{Conclusion}
%
%	\subsubsection{Method I:Connected corner approach} % cCorner
%
%	\subsubsection{Method II: histogram based approach}









\section{Interactive demo}
	osgviewer demo of 3d point cloud and images
	demo of matlab progje


\end{document}

