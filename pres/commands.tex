% PACKAGES
\usepackage{epsfig}
\usepackage{graphicx, subfigure}
%\usepackage{float}


% COMMANDS

\newcommand{\fig}[1]{
	\begin{figure}[!ht]
	\centering
	%\includegraphics[=#3]{../img/#1}
	\includegraphics[width=300px]{../img/#1}
	%\caption{#2}
	\label{fig:#1}
	\end{figure}
}
\newcommand{\figCustom}[2]{
	\begin{figure}[!ht]
	\centering
	%\includegraphics[=#3]{../img/#1}
	\includegraphics[#2]{../img/#1}
	%\caption{#2}
	\label{fig:#1}
	\end{figure}
}
\newcommand{\figs}[1]{
	\begin{figure}[!ht]
	\centering
	\includegraphics[width=100px]{../img/#1}
	%\caption{#2}
	\label{fig:#1}
	\end{figure}
}

\newcommand{\figsHor}[2]{
	\begin{figure}[!ht]
	\centering
	\subfigure[]{
		\includegraphics[width=120px]{../img/#1}
	}
	\subfigure[]{
		\includegraphics[width=120px]{../img/#2}
	}
	\end{figure}
}

\newcommand{\fignocaption}[3]{
	\begin{figure}[!ht]
	\centering
	\includegraphics[scale=#3]{img/#1}
	\label{fig:#1}
	\end{figure}
}

