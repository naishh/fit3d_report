%todo
\section{Conclusion}
% dit gaat samenvattende conclusie van ch5 en ch6 worden!! 
%conclusie is geen abstract!
In this thesis we annotated urban scenes using skyline detection and window
detection.\\

We proposed a skyline detection algorithm that is simple has a low complexity
and works without any user interaction.  Under the assumption that the skyline
is the upper edge, the method works well.  A set of upper edges should be seen
as potential hypothesis which are evaluated using additional features like color
and height variation.\\

Furthermore we proposed two window detection methods that both detect at least
99\% of the windows.  We conclude that for uncalibrated scenes the connected
corner approach performed best as it is robust to variation in window type and
viewing angle.  For calibrated scenes we conclude that 
interpreting the amount of Hough lines is a strong approach
towards determination of the window alignment. 
We proposed several alignment methods and conclude that they
perform best if we combine them.
Furthermore we developed two classification method. The winner is based on the
derivative of the Houghlines Histogram function.  \\

In this project we retrieved important semantics of an urban scene: a
full 3D construction of the building and the extraction of his windows.
Although this is very little in comparison to what humans perceive, we made a
valuable start towards human-like interpretation of urban scenes.\\

The applications grow in the number of correct derived semantics. With just the
extracted 3D model and the extracted windows we can recognize buildings,
build 3D city models, monitor building deformation and add augmented reality
to scenes.  Even more application would rise if we add more semantics like
house numbers, doors, trees, cars or even (your stolen?) bicycle. \\

The skyline is the limit.

